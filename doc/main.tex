%% Default is IEEE. Add "APA" as an option to use APA
% add twoside for final
% singlespace
% UoMdraft
\documentclass[12pt,APA]{uomthesis}


\usepackage{packages/lgrind}
%\usepackage{lgrind}
\usepackage{cmap}
\usepackage[T1]{fontenc}
\pagestyle{plain}
\usepackage{graphicx}
\usepackage{amsmath}
\usepackage{tabularx}
%\renewcommand\tabularxcolumn[1]{m{#1}}% for vertical centering text in X column
\newcolumntype{Y}{>{\centering\arraybackslash}X}
\usepackage{enumerate}
\usepackage{multirow}
\usepackage{caption}
\usepackage{subcaption}
\usepackage{enumitem}
\usepackage{mathdots}
% \usetikzlibrary{fadings}
\usepackage{bbm}
\captionsetup{compatibility=false}
\DeclareMathOperator*{\argmax}{argmax}
\usepackage{tikz}
\usepackage{epsfig}
\usepackage[linesnumbered,ruled,vlined]{algorithm2e}
\usepackage{makecell}
\usepackage{hyperref}
\usepackage{rotating}
\hypersetup{
colorlinks   = true,
citecolor    = blue,
urlcolor    =  blue
}
\usepackage{url}
\usepackage{longtable}
\usepackage{pdflscape}

\begin{document}
% Update the following variables. Doing so will auto-generate the first three pages:
% 1. Cover Page
% 2. Title Page
% 3. Declaration Page
\title{Revolutionizing Flow Rate Prediction in Oil and Gas Industry: Dynamic Virtual Flow Meter Empowered by Transformers}
\student{239171B}{Priyanka Perera}
%\student{}{} % If there are multiple student authors (e.g., undergraduate project), just keep adding \student{}{} variables. The class will automatically handle it
\degree{Master's in Data Science and Artificial Intelligence} % PhD/MPhil/Master’s 
\department{Department of Computer Science and Engineering}
\type{Dissertation} % Thesis/Dissertation
\degreemonth{December}
\degreeyear{2023}
\faculty{Faculty of Engineering}
\keywords{Virtual Flow Meter, Oil and Gas Industry, Transformers}
\supervisorName{Dr. Uthayasanker Thayasivam}
\supervisorName{Dr. Sandareka Wickramanayake}


\maketitle


\unNumChapter{Dedication}

To my beloved wife, your unwavering love, encouragement, and sacrifices have been the cornerstone of my academic journey. Your constant support gave me the strength to pursue my dreams and overcome challenges along the way.

To my dedicated supervisors, I am indebted to your guidance, knowledge and patience. Your mentorship has shaped my academic growth and instilled in me a passion for knowledge that goes beyond the lectures. Your dedication to teaching and learning has been a beacon of inspiration throughout my studies.

\unNumChapter{Acknowledgement}
I extend my heartfelt gratitude to Cognite AS for their invaluable support and expertise in offering profound domain knowledge throughout the course of my research. Their guidance has been instrumental in shaping the direction and depth of my study. The insights and perspectives shared by the team at Cognite AS have significantly enriched the quality and relevance of this research endeavor. I am deeply appreciative of the time, resources, and commitment extended by Cognite AS in providing invaluable assistance, which has been pivotal in navigating the complexities of this study. Their dedication to advancing knowledge and innovation within their domain has been a constant source of inspiration.

I extend my sincere appreciation to Dr. Sandareka and Dr. Thayasivam from the University of Moratuwa for their invaluable supervision, unwavering support, and insightful guidance throughout the course of my research. Their expertise, encouragement, and dedication have been instrumental in shaping the trajectory and depth of this study.


\begin{abstractpage}
The accurate measurement of oil flow rates is critical for optimizing oil well production. Traditional physical flow meters are often costly, require regular maintenance, and are prone to failure in harsh operating conditions. This research proposes a novel approach to address these challenges by leveraging temporal fusion transformer neural networks for virtual flow metering in oil well production. The transformer architecture, known for its success in natural language processing tasks, is adapted to capture the complex relationships and dependencies in the oil flow data using TFT. This research presents a detailed methodology for training the virtual flow meter using historical production data and demonstrate its performance on a real data-set. Additionally, this research explore the potential of transfer learning techniques to enhance the generalization capabilities of the virtual flow meter across different well configurations and operating conditions. The findings of this research contribute to the advancement of oil well production optimization by providing an accurate, cost-effective, and reliable alternative to traditional flow meters.
\end{abstractpage}


% This prints the Table of Contents. Do not change this block.
\begin{ToC}
\makeatletter
    \@starttoc{toc}
 \makeatother
\end{ToC}

% This prints the List of Figures, if there is at least one figure in your document, set up according to the given guidelines.
\iffigures
    \begin{LoF}
        \makeatletter
            \@starttoc{lof}
        \makeatother
    \end{LoF}
\fi


% This prints the List of Tables, if there is at least one table in your document, set up according to the given guidelines.
\iftables
    \begin{LoT}
    \makeatletter
        \@starttoc{lot}
    \makeatother
    \end{LoT}
\fi

% This prints the List of Abbreviations, if there is at least one abbreviation in your document, set up according to the given guidelines.
\ifabbrs
    \printacronyms
\fi

% This prints the List of Appendices, if there is at least one abbreviation in your document, set up according to the given guidelines.
\ifappens
    \begin{LoAp}
    \makeatletter
        \@starttoc{apc} % Print List of Appendices
    \makeatother
    \end{LoAp}
\fi

% This marks the start of your chapters
\startContent


\chapter{Introduction}
\label{Sec:Intro}

Oil and gas production is a complex process that involves the extraction of hydrocarbons from underground reservoirs. Oilfield is a geographical area that encompasses one or more oil or gas reservoirs equipped with the necessary infrastructure for exploration, extraction and production. The fluid mixture flowing out of the well consists of oil, natural gas and water. This mixture of flow is called the multi-phase flow. The flow rates of oil, gas and water reflect the performance of the well. The Petroleum Engineers need to make informed decisions regarding the production strategies based on the performance of the wells. Through accurate predictions of flow rates, petroleum Engineers can diagnose production inefficiencies and constraints, and implement solutions to optimize production and project future investments. Physical flow meters and \abbr{Virtual Flow Meter}{VFM} are used to predict flow rates in the oil wells. Physical flow meters are often very expensive, require regular maintenance, and are prone to failure in harsh operating conditions. Physical flow meters use in-built sensor data and physics based mathematical equations to predict the real-time flow rates. Virtual flow meters in contrast, can use physics based mathematical models, machine learning based models or hybrid models to predict the flow rates \citep{bikmukhametov2020first}. 

\section{Oil Well Configuration}
The configuration of sensors in an oil well can vary depending on the specific needs of the well, but generally, they are strategically placed to monitor various aspects of the drilling, production, and safety of the well. The wellbore, often simply referred to as the "borehole" or "hole," is the hole or opening that is drilled or dug into the earth's surface during the process of drilling an oil, gas, water, or geothermal well. It's a cylindrical shaft that extends from the surface down into the subsurface layers of the earth. Typically, oil wells are installed with temperature, pressure and valve sensors in well-down-hole (DH), well-head (WH) and downstream choke (DC \ref{fig:well-sensors}. The "well down hole sensors" refer to instrumentation placed at the bottom of an oil or gas well, often within the drilling equipment or down-hole tools. These sensors are designed to gather crucial data about the conditions, characteristics, and behavior of the wellbore and the surrounding subsurface formations. They typically measure parameters such as pressure, temperature and fluid properties, and sometimes even seismic data. These sensors play a vital role in monitoring and optimizing various aspects of well operations. By providing real-time data from deep within the well, they help in making informed decisions related to drilling, production, reservoir management, and maintenance. Well down-hole sensors aid in assessing reservoir performance, optimizing production rates, identifying potential issues or anomalies, and ensuring the overall efficiency and safety of the well operation. Wellhead sensors are instruments installed at the wellhead—the surface area where the wellbore meets the surface equipment. These sensors are crucial components in oil and gas operations as they monitor and collect various data points related to the well's performance, production, and safety. Some common types of wellhead sensors include pressure sensors, temperature sensors, and level sensors. The choke sensors, also known as choke position sensors or choke valve position sensors, are devices used in oil and gas production systems to monitor the position and movement of choke valves. Choke valves are crucial components in controlling the flow of fluids such as oil, gas, and water from the wellhead into the production system.

\begin{figure}[!htb]
\uomFig{Oil well sensors}{fig:well-sensors}{\includegraphics[width=0.9\textwidth]{images/well-sensors.png}}{Explanations to clarify information in the table (if needed).}
\end{figure}

\begin{table}[!htb]
\uomTable{Oil well sensors}{tab:well-sensor}{\begin{tabularx}{0.9\textwidth}{lYYYY} 
\hline
Sensor & Description & Unit \\
\hline
DHP & Down-hole pressure & Pa \\
DHT & Down-hole temperature & C \\
WHP & Well-head pressure & P \\
WLT & Well-head temperature & C \\
CHK & Choke open ratio & \% \\
\hline
\end{tabularx}}{Explanations to clarify information in the table (if needed).}
\end{table}

\section{Types of Flow}
The flow of the fluid mixture can be in steady state or dynamic. The steady state flow occurs when the multi-phase flow moves through the well and reservoir under conditions where the flow rates and pressure differentials remain relatively constant over time. Steady-state flow is defined as a flow condition under which the pressure at any point in the reservoir remains constant over time \citep{dake1978developments}. Dynamic flow refers to the fluid movement in the well and reservoir under conditions where flow rates, pressure gradients, and other flow parameters are subject to continuous changes and fluctuations.

\section{Physical Flow Meters}
Physical flow meters encompass a diverse range of instruments designed to measure the rate of fluid flow across various industries and applications. These meters operate based on physical principles, employing mechanisms like pressure differentials, velocity measurement, displacement, mass calculation, or volumetric methods. Their versatility lies in their ability to cater to a wide array of fluids, from gases and liquids to multi-phase mixtures. 

\subsection{Single Phase Flow Meter}
Traditional single-phase metering systems necessitate the complete separation of the phases within the well streams before the measurement. In production metering, this separation typically occurs naturally at the exit of a conventional processing facility. These plants are designed to amalgamate various well streams at one end and deliver stabilized individual phases for transport at the other end. Typically, single-phase metering systems offer top-notch accuracy in measuring hydrocarbon production \citep{corneliussen2005handbook}.

\subsection{Multi Phase Flow Meter}
\abbr{Multi Phase Flow Meters}{MPFM} in the oil industry is a specialized device designed to measure the flow rates of oil, water, and gas as they are produced from a well. The need for multi-phase flow metering arises when it is necessary or desirable to meter well stream(s) upstream of inlet separation and/or commingling. Multi-phase flow measurement technology may be an attractive alternative since it enables measurement of unprocessed well streams very close to the well. The use of MPFMs may lead to cost savings in the initial installation. However, due to increased measurement uncertainty, a cost-benefit analysis should be performed over the life cycle of the project to justify its application. It operates by determining the individual phase fractions (oil, water, gas) and their flow rates within a single pipeline or well-bore. MPFMs are extremely expensive and require frequent calibrations due to the drift in sensors and changes in fluid properties. MPFMs also considered less accurate outside the normal operational range. \citep{meribout2020multiphase}.

\begin{figure}[!htb]
\uomFig{Multi phase flow meter}{fig:mpfm-1}{\includegraphics[width=0.9\textwidth]{images/mpfm-1.png}}{Explanations to clarify information in the table (if needed).}
\end{figure}

\section{Virtual Flow Meters}
Over the past two decades, the evolution of VFM has led to the creation of diverse techniques for estimating multi-phase flow rates using field data. This progress has seen the emergence of commercial VFM systems from various companies, widely adopted by oil and gas operators globally. Presently, some methods seek to enhance flow rate prediction accuracy, while others, though not currently employed in the industry, hold promising potential for advancing VFM development in the future. When it comes to modeling approaches, three primary categories of Virtual Flow Metering can be identified

\subsection{Physics-based VFM}
Physics based VFMs are also known as first principles VFM systems. Physics based VFMs dominate the industry as they've undergone significant development over half a century to comprehensively outline each facet of this approach. This extensive effort has yielded a strong grasp of mechanistic modeling within production systems, fluid properties, and optimization methods. Consequently, first principles modeling stands as a dependable method to depict overall production system behavior and especially multi-phase flow phenomena.

\subsection{ML based VFM}
\abbr{Machine Learning}{ML} based modeling involves analyzing system data to uncover connections between input and output variables without precise knowledge of the system's physical behavior. This method proves advantageous in bypassing intricate physical modeling, especially for systems like multi-phase flows in pipes, where exact numerical solutions can be challenging. It relies on experimental or industrial data to grasp the system's behavior directly and attempts to learn the underlying relationships from this data.

\section{Research Problem}
\label{Sec:Problem}

Well flow rate prediction is a challenging problem due to the complex design of the well and the complicated physics of the fluid mixture flowing out of the well. These challenges affect accurate predictions regardless of the type of flow meters used. Physical flow meters are the most common method of measuring oil and gas well rates. They are however extremely expensive to install and maintain. The accuracy of the physical flow meters gradually reduces over time and requires re-calibration. VFMs are a growing alternative to physical flow meters. VFMs are less expensive than physical flow meters and can be used when physical flow meters are not financially viable. VFMs are typically less accurate than physical flow meters, and they they are often complex to develop and implement. The hybrid VFM models which use physics and machine learning have become increasingly popular since 2018. Increase usage of \abbr{Neural Networks}{NN} including \abbr{Recurrent Neural Networks}{RNN} and \abbr{Long Short Term Memory}{LSTM} also seen for developing both physics and ML based models \citep{franklin2022physics}.

Although many types of neural networks have been evaluated in literature, author at the time of writing did not discover any experimentation or implementation of transformers for VFMs. Transformers \citep{vaswani2017attention}, originally designed for natural language processing tasks, have shown remarkable performance in capturing complex and long range dependencies in sequential data. \abbr{Temporal Fusion Transformer}{TFT} \citep{lim2021temporal} a type of a transformer network, on a variety of real-world data-sets have demonstrated significant performance improvements over existing benchmarks for predicting time-series data. The author therefore argues that developing a VFM based on promising transformers holds potential for accurate and cost-effective flow rate prediction leading to greater production optimization and profits.

\section{Research Objectives}
\label{Sec:Objectives}

\begin{enumerate}
  \item Investigate the feasibility and effectiveness of the TFT architecture in capturing the complex relationships and dependencies present in oil flow data.
  \item Develop a virtual flow meter based on TFT neural networks to accurately estimate oil flow rates in real-time for oil well production optimization.
  \item Explore the use of historical production data to train the virtual flow meter and establish its accuracy and reliability compared to traditional flow meters.
  \item Evaluate the performance of the TFT-based virtual flow meter in various well configurations and operating conditions, ensuring its adaptability and versatility.
  \item Investigate transfer learning techniques to enhance the generalization capabilities of the virtual flow meter across different oil well production settings.
  \item Compare the performance of the proposed virtual flow meter with existing flow metering methods, quantifying the advantages in terms of cost, accuracy, and maintenance requirements.
  \item Provide recommendations and guidelines for implementing and deploying the TFT-based virtual flow meter in real-world oil production scenarios.
  \item Assess the scalability and potential for integration with existing oil production systems, considering computational requirements and data management aspects.
  \item Contribute to the broader field of machine learning in industrial applications by showcasing the applicability of TFT neural networks in optimizing oil well production processes.
\end{enumerate}


\chapter{Literature Review}

Measuring flow rates in wells faces several challenges due to the dynamic and complex nature of well environments. Wells produce a mixture of oil, gas, water, and sometimes solids. This multi-phase flow makes it difficult to directly measure individual phase flow rates accurately.  For Virtual Flow Metering (VFM) applications, ML based modeling involves collecting data such as down-hole and wellhead pressures and temperatures, choke opening values, ESP parameters, and corresponding oil, gas, and water flow-rate measurements. These measurements can originate from various sources, like well test data or MPFM. The ML based model can serve as a backup metering system for individual wells if MPFMs are installed for each wellhead. Alternatively, if one MPFM serves multiple wells, its data aligns with well test or separator data, enabling flowrate measurements per well according to testing schedules. Post-training and validation, the ML based model functions as a standalone VFM system.

The area that contains the deposit of oil and/or natural gas is called the oil reservoir. There can be multiple wells connected to the same reservoir. Fluid flow in the piping network in a oil well is a multi-phase flow containing oil, gas and water. Piping network in a oil and gas well can be a combination of horizontal, vertical or inclined pipes. Therefore the flow of fluid in the piping can occur at any direction including upward and downward flows. Flow of fluid in the well can transit between the steady state and unsteady state. The steady state refers to the situation when the flow rates of the phases of the flow remains approximately uniform over time. Unsteady state on the other hand refers when the flow rates of individual phases rapidly varies over time.  There are two mechanisms of measuring the flow rates of the oil well. They are physical flow meters and virtual flow meters (VFM). They physical flow meters can be of two types. They are multi-phase flow meter and single phase flow meter. Even in the physical flow meters, flow rates are calculated indirectly using physics based mathematical model. Multi-phase flow meter measure the flow rates of different phases simultaneously without separation. Multi-phase flow meters can provide continuous measurement of flow rates at sensor locations with decent accuracy but they are considered extremely expensive and requires regular calibration. Multi-phase flow meters are however considered to be less accurate outside of normal operational ranges. Physical flow meters in general also have the limitation of not being able to back-fill historical flow rates prior to the installation of the instrument. Single phase flow meters in contrast measure the flow rate of each phase individually, therefore multi-phase flow should be separated through a separator prior to feeding the single phase flow meter. Single phase flow meters are considered more accurate than multi-phase flow meter. Although not as advanced as multi-phase flow meters, the single phase flow meters are also considered expensive and its usage is extremely costly due to production deferment. This is because the well has to be disconnected from the production and needs to be connected with the separator prior to feeding the single phase flow meter. Single phase flow measurement is often called a well test. Some wells cannot be measured from a well test because the distance from the well to the separator can be very long. Both VFMs and physical flow meters measure the flow rates indirectly. Physical flow meters bring their own set of sensors whereas VFMs use the existing sensors in the well. Physical flow meters use physics based mathematical models to calculate the flow rates whereas VFM can use physics models and/or ML based models for prediction. VFMs, therefore can be categorized into 3 types; physics based, ML based and hybrid flow meters. Physics based VFMs are also called first principle VFMs or physics simulator based VFMs. Physics based VFMs are generally considered expensive due to licensing costs of the simulator and slow in prediction due to processing of high complexity mathematical models. Physics based VFMs require very accurate descriptions of the well, fluids, trajectory, production choke and installation parameters for modeling the simulator based on the principles of physics. ML based VFMs on the other hand are purely data driven. Hybrid VFMs implement combined physics based and ML based models. All 3 types of VFMs rely on pressure and temperature sensor data in the well. ML based VFM and hybrid VFM also rely on physical well-test measurements for model training. More the well-test measurements available, higher the accuracy of the model within the operating conditions of the well-test measurements. They are however said to have higher degree of uncertainty outside the operating conditions of the well-test measurements. Oilfields are required for perform well-tests at some point based on regulatory requirements of the jurisdictions. It is however not economically feasible to run regular well-tests as they cause production deferment. All types of VFMs have the limitation of being restricted to steady state, meaning that if there are rapid transient changes in the flow, VFMs will not predict them accurately. Therefore VFMs cannot predict startup, shutdown and other transient scenarios. ML and hybrid VFMs also cannot accurately handle advanced what-if predictive scenarios. Physics based VFMs in contrast can accurately handle advanced what-if predictive scenarios.

In a typical well, pressure and temperature sensors are deployed in the well down-hole (WBH), well-head (WH) and downstream of the choke (DC). The choke is used to control the produced flow in the well for situational requirements. The six pressure and temperature sensors along the the choke position is generally used as independent variables for predicting the flow rates in ML based VFMs.

\citep{al2017radial} introduces a radial basis function network designed to create a virtual flow meter (VFM) specifically for estimating gas flow rates within multiphase production lines. Validating the model with real well test data demonstrates its outstanding performance and ability to generalize effectively. Furthermore, the paper delves into the importance of bottom-hole and choke valve measurements in ensuring precise predictions. This proposed VFM model presents a potentially appealing and cost-efficient solution for real-time production monitoring needs while simultaneously reducing operational and maintenance expenses.

\citep{al2017development} introduces a method for estimating phase flow rates in oil and gas production wells using readily available measurements. By overcoming the limitations of traditional metering facilities, this system offers a cost-effective way to monitor production in real time. It not only cuts operational and maintenance expenses but also serves as a reliable backup to multi-phase flow meters. The technique involves creating a "soft sensor" through a feed-forward neural network. To ensure accuracy without excessive complexity, the system uses K-fold cross-validation and an early stopping technique to regulate generalization and network intricacy. Validation using real well test data demonstrates the sensor's effectiveness, evaluated through cumulative deviation and flow plots. Their results indicate promising performance, with an average error of approximately 4\% and less than 10\% deviation for 90\% of the samples.

The hybrid VFM proposed by \citep{ishak2022virtual} uses ensemble learning to develop the data driven model by incorporating multiple ML models. The data driven model is then combined with the physics model using a combiner. They have achieved a 50\% improvement in performance using the combiner compared to their stand-alone performance of physics based and ML based VFMs. 

\citep{al2018virtual} introduces a VFM system employing ensemble learning tailored for fields with limited data from a shared metering setup. The method creates diverse NN learners by manipulating training data, NN structure, and learning approach. Using \abbr{Adaptive Simulated Annealing}{ASN} optimization, it selects the best subset of learners and an optimal combining strategy. Assessment using real well test data shows exceptional performance, with average errors of 4.7\% and 2.4\% for liquid and gas flow rates respectively. The accuracy of their VFM was further confirmed through cumulative deviation analysis, where predictions fall within a maximum deviation of ±15\%. Comparisons with standard bagging and stacking techniques demonstrate significant enhancements in both accuracy and ensemble size. The proposed VFM system offers ease in development and maintenance compared to traditional model-driven VFMs, requiring only well test samples for model tuning. It is anticipated that this developed VFM can complement physical meters, improve data consistency, aid in reservoir management and flow assurance, ultimately resulting in more efficient oil recovery and reduced operational and maintenance costs.

\citep{andrianov2018machine} have successfully developed a LSTM-RNN based model not only to predict multi-phase rates at present but also to predict future flow rates as a time-series. This is because unlike feed-forward NN, which process input data in a one-way, LSTM-RNN are designed to process sequential data, such as time series. \citep{andrianov2018machine} has achieved the best accuracy when the lengths of the input and output sequences to LSTM are equal. 

\citep{grimstad2021bayesian} develops a ML based VFM, implemented on \abbr{Bayesian Neural Network}{BNN}. BNN provide a probabilistic distribution over weights and predictions unlike traditional neural networks, which produce point estimates for weights and predictions. They have trained the model on a large and heterogeneous data-set, consisting of 60 wells across five different oil and gas assets. The predictive performance is analyzed on historical and future test data, where an average error of 4\%–6\% and 8\%–13\% is achieved for the 50\% best performing models, respectively. 

\citep{franklin2022physics} proposes a hybrid VFM combining \abbr{Physics-informed Neural Networks}{PINN} based physics model and LSTM-RNN based ML model. Their resulting hybrid model is capable of predicting the average flow rate some time-steps ahead using the measurements available in the oil well information system. Unlike the ML based LSTM-RNN model proposed by \citep{andrianov2018machine}, the system proposed by \citep{franklin2022physics} is hybrid model using NN for physics based model as well as the ML based model. 

\citep{song2022intelligent} employs Back Propagation (BP) neural network, Long Short-Term Memory (LSTM) network, and Random Forest algorithm to develop an intelligent ML based model for virtual flow meters in oil and gas development. Their data-set is constructed using actual data from two oil wells in an offshore oil field in the South China Sea. Among the three models, the LSTM model has demonstrated the highest accuracy, with a Mean Absolute Error (MAE) of 3.9\%. LSTM has also demonstrated the highest stability and requires a moderate amount of data volume. BP network on the other hand have exhibited the lowest accuracy, with a MAE of 12.1\%, as well as the lowest stability. BP however has shown the smallest data volume requirement. The Random Forest model has shown moderate accuracy, high stability, but has required the highest data volume.

\citep{muchsin2023virtual} proposed a VFM model using a time series-time delay artificial neural network technique in predicting the multi-phase flow rate accurately with the best average discrepancy of Qgas (6.0\%), Qoil (-16.4\%), and Qwater (-2.4\%) compared with actual measurement by MPFM. The training of the VFM model, which takes less than 7 minutes and has acceptable values for MSE, R, and MAPE, demonstrates that this method is reliable enough to be used in the oil and gas industries, which often require conducting dozens of individual well tests in their daily activities. The utilization of data measurement of existing well orifice meter combined with data measurement from choke valve and well head as parameter data input of the network has proven effective to improve the accuracy of the VFM model. In addition, the greater the number of data used in training phase determines the accuracy performance of the VFM model. However, concerning that this study only uses limited well reference data, it is recommended to implement this VFM model on other wells in different fields to validate its performance accuracy during well testing.

\citep{nemoto2023cloud} introduces a cloud-based VFM that combines physics and data techniques to accurately estimate water production per well within a gas field. By integrating physics-based models tailored for high gas volume fraction gas-liquid flows in the well-bore and employing a data-driven approach to fine-tune these models with actual well test data, this hybrid method ensures precise real-time estimations. Its adaptability to evolving well performance and increased water production creates a self-calibrating solution, ensuring ongoing accuracy and relevance despite changing production and well conditions. Their VFM system demonstrates substantial alignment with well test data under steady-state conditions, affirming its reliability. Operating remotely through a cloud-based Cognite DataOps platform, this system conducts calculations and stores results, facilitating continual access to live sensor data for other applications or visualization via a web interface. Utilizing existing sensors within the wells, the VFM system offers cost efficiency by minimizing both initial investment and operational expenses in comparison to installing multi-phase flow meters or separators.

Hybird VFM models have become increasingly popular since 2018 and increase usage of NNs seen for developing both physics and ML based models. Although many types of NNs have been evaluated, author at the time did not discover Transformer based NN for implementing VFM. Transformers, with the ability to capture long-range patterns in sequential data, can be applied to time-series data obtained from sensors.

\begin{longtable}[!htb]{p{0.2\textwidth}p{0.1\textwidth}p{0.2\textwidth}p{0.35\textwidth}}
%\uomTable{Summary of literature}{tab:summary-literature}{\begin{tabularx}{0.9\textwidth}{lYYYY}
\uomTable{Summary of literature}{tab:summary-literature}{}{}
\hline
Research & VFM Type & ML Model & Evaluation Metrics \\
\hline
\endfirsthead
\multicolumn{2}{l}%
{{\tablename\ \thetable{} -- continued}} \\
\hline
Research & VFM Type & ML Model & Evaluation Metrics \\
\hline
\endhead % Header for subsequent pages
\cite{al2017radial} & ML & ANN & Oil) \(R^{2} = 0.965, RMSE = 1.24899, MAPE = 4.22\) Gas) \(R2 = 0.954, RMSE = 1.35277, MAPE = 2.27\) \\
\cite{al2017development} & ML & RBF & \(R^{2} = 0.93978, RMSE = 1.334, MAPE = 6.16\) \\
\cite{andrianov2018machine} & ML & LSTM, RNN & Not reported \\
\cite{al2018virtual} & ML & Ensemble NN and ASN & \(ANN: RMSE = 0.0585; STDEV = 0.0046; MAPE = 4.7\) \(ASA: RMSE = 0.0442, STDEV = 0.0036, MAPE = 2.35\) \\
\cite{dutta2018modeling} & ML & ANN FPA & \(RMSPE = 0.75, ARPE = 99.25\) \\
\cite{hansen2019multi} & ML & ANN & $\text{\(a) R = 0.993, AAPE = 8.39\)}$ $\text{\(b) R = 0.995, AAPE = 6.36\)}$ \\
\cite{alrumah2019new} & ML & ANN, LSSVM, SIMPLEX & \(ANN: R2 = 0.9292, RMSE = 863.98, AARPE = 22.06\) \(LSSVM: R^{2} = 0.9477, RMSE = 719.6, AARPE = 21.5\) \(SIMPLEX: R^{2} = 0.885, RMSE = 1067, AARPE = 26.8\) \\
\cite{bikmukhametov2020combining} & ML & MLP, LSTM, GB & \(Oil:\) \(MLP: RMSE = 0.0458; LSTM: RMSE = 0.0476\) \(GB: RMSE = 0.0463\) \(Gas:\) \(MLP: RMSE = 0.0328, LSTM: RMSE = 0.278\) \(GB: RMSE = 0.0367\) \\
\cite{hotvedt2020developing} & Hybrid & Not reported & \(RMSE = 15, MAE = 8\) \\
\cite{marfo2021predicting} & ML & ANN & \(R = 0.9966, MAPE = 3.18\) \\
\cite{grimstad2021bayesian} & ML & BNN & \\
\cite{ishak2022virtual} & Hybrid & Ensemble model & \\
\cite{song2022intelligent} & ML & BPNN, LSTM, Random Forest & \\
\cite{franklin2022physics} & Hybrid & PINN, LSTM, RNN & \(10^{-4} \leq MSE \leq 10^{-3}\) \\
\cite{muchsin2023virtual} & ML & TSTD-ANN & \(MSE\leq 1500, R^{2} \geq 0.95, MAPE \leq 65\) \\
\cite{nemoto2023cloud} & Hybrid & Linear Regression & M1) \(MRPE = 12.64, MAPE = 20.61, R^{2} = 0.77\) M2) \(MRPE = 0.43, MAPE = 17.43, R^{2} = 0.80\) M3) \(MRPE = -0.23, MAPE = 20.10, R^{2} = 0.64\) \\
\hline
\end{longtable}


\chapter{Methodology}
The author proposes a novel approach to address the accuracy challenges by leveraging the capabilities of the TFTs and exploring transfer learning techniques to generalize across different well configurations and operating conditions. In the domain of multi-horizon forecasting, there's often a complex mix of various inputs including time-variant and time-invariant inputs. However, understanding how these inputs interact with the intended prediction target is typically unknown beforehand. \citep{lim2021temporal} introduce TFT, an innovative architecture that merges high-performance multi-horizon forecasting with the ability to unravel the temporal dynamics behind these forecasts. TFT achieves this by employing a combination of recurrent layers for localized processing and interpretable self-attention layers to capture long-term dependencies. To comprehend temporal relationships across various time scales, TFT employs specialized components to identify crucial features and a sequence of gating layers to minimize unnecessary components. This approach enables TFT to excel in diverse scenarios. Their experiments with real-world data-sets, we showcase significant performance enhancements compared to existing benchmarks. 

\begin{enumerate}
    \item Numerous deep learning methods have been proposed for timeseries forecasting, but they're often seen as 'black-box' models. They perform well in forecasting but lack transparency in revealing how they leverage the entire spectrum of inputs present in real-world scenarios
    \item Second item
\end{enumerate}

\section{Data-set}
The initial phase of this research involves the meticulous collection and preprocessing the Multi Phase Flow Meter (MPFM) sensor data. The MPFM dataset, sourced from wells with operational sensors, provides a wealth of historical information encompassing oil, gas, and water flow rates, among other pertinent parameters. This information undergoes a rigorous preprocessing phase, where missing values are addressed, outliers are identified and managed, and normalization techniques are applied to ensure data uniformity and integrity. Simultaneously, the Well Test Data, retrieved from wells lacking MPFM sensors, is curated, focusing on extracting parameters correlating with flow rates. Similar preprocessing steps are undertaken, ensuring alignment with the features derived from the MPFM data-set.

\section{Model}
The crux of this research lies in the implementation and training of the Temporal Fusion Transformer (TFT) neural network using the preprocessed MPFM data-set. The architecture of the TFT is meticulously crafted to accommodate the intricacies of multi-phase flow rate predictions. Through an iterative process, hyper-parameters are fine-tuned to optimize the model's performance. The MPFM dataset is divided into distinct subsets for training, validation, and testing purposes. Subsequently, the TFT model undergoes comprehensive training using the MPFM sensor data to forecast oil, gas, and water flow rates. Rigorous validation protocols are employed to prevent overfitting and ensure the model's robustness and reliability.

Upon establishing a proficiently trained TFT model using MPFM data, the research extends to wells devoid of MPFM sensors, necessitating a transfer learning paradigm. Feature alignment becomes paramount, identifying and reconciling common features between the MPFM and well test datasets to enable transfer learning. Leveraging the pre-trained TFT model, a transfer learning approach is initiated, fine-tuning the model using the well test data to adapt and cater to the unique conditions of the new wells lacking sensor technology. Thorough validation of the transfer-learned model is conducted, utilizing separate test sets from wells without MPFM sensors to assess its performance and predictive capabilities.

\section{Evaluation}
The evaluation of model performance constitutes a pivotal phase of this research endeavor. Performance metrics such as Root Mean Square Error (RMSE), Mean Absolute Error (MAE), or R-squared values are employed to gauge the accuracy and efficacy of flow rate predictions. Comparative analyses between predicted and actual flow rates provide insights into the model's accuracy and reliability. Cross-validation procedures are rigorously employed to ensure the robustness and generalizability of the trained models across various datasets and scenarios.

\begin{references}
    \bibliography{references} % argument is your bibliography database(s)
\end{references}

\begin{appendices}

\appendiceC{How to Extract Bibliographic Entries}{App:GScholar}

\end{appendices}

\end{document}



