%% Default is IEEE. Add "APA" as an option to use APA
% add twoside for final
% singlespace
% UoMdraft
\documentclass[12pt,APA]{uomthesis}
\usepackage{packages/lgrind}
%\usepackage{lgrind}
\usepackage{cmap}
\usepackage[T1]{fontenc}
\pagestyle{plain}
\usepackage{graphicx}
\usepackage{amsmath}
\usepackage{tabularx}
%\renewcommand\tabularxcolumn[1]{m{#1}}% for vertical centering text in X column
\newcolumntype{Y}{>{\centering\arraybackslash}X}
\usepackage{enumerate}
\usepackage{multirow}
\usepackage{caption}
\usepackage{subcaption}
\usepackage{enumitem}
\usepackage{mathdots}
% \usetikzlibrary{fadings}
\usepackage{bbm}
\captionsetup{compatibility=false}
\DeclareMathOperator*{\argmax}{argmax}
\usepackage{tikz}
\usepackage{epsfig}
\usepackage[linesnumbered,ruled,vlined]{algorithm2e}
\usepackage{makecell}
\usepackage{hyperref}
\usepackage{rotating}
\usepackage{textcomp}
\hypersetup{
colorlinks   = true,
citecolor    = blue,
urlcolor    =  blue
}
\usepackage{url}
\usepackage{longtable}
\usepackage{pdflscape}
\usepackage{amsfonts}
\usepackage{booktabs}
\usepackage{array}

\begin{document}
% Update the following variables. Doing so will auto-generate the first three pages:
% 1. Cover Page
% 2. Title Page
% 3. Declaration Page
\title{Physics-Informed Residual Learning Architecture for Virtual Flow Metering in the Oil and Gas Industry}
\student{239171B}{Priyanka Perera}
%\student{}{} % If there are multiple student authors (e.g., undergraduate project), just keep adding \student{}{} variables. The class will automatically handle it
\degree{Master's in Data Science and Artificial Intelligence} % PhD/MPhil/Master’s 
\department{Department of Computer Science and Engineering}
\type{Dissertation} % Thesis/Dissertation
\degreemonth{December}
\degreeyear{2025}
\faculty{Faculty of Engineering}
\keywords{Virtual Flow Meter, Oil and Gas Industry, Physics Informed Machine Learning}
\supervisorName{Dr. Uthayasanker Thayasivam}


\maketitle


\unNumChapter{Dedication}

To my beloved wife and parents, your unwavering love, encouragement, and sacrifices have been the cornerstone of my academic journey. Your constant support gave me the strength to pursue my dreams and overcome challenges along the way.

\unNumChapter{Acknowledgement}
I extend my sincere appreciation to Dr. Uthayasanker Thayasivam from the University of Moratuwa for the invaluable supervision and insightful guidance throughout the course of my research.


\begin{abstractpage}
Accurate estimation of multiphase flow rates is essential for effective production monitoring and reservoir management in the oil and gas industry. Although Multiphase Flow Meters provide direct measurements of oil, gas, and water rates, their high cost and operational complexity limit widespread deployment, leading to sparse and irregular flow rate observations. Virtual Flow Metering offers a software-based alternative; however, conventional physics-based VFM models are often constrained by simplifying assumptions and parameter uncertainty, while purely data-driven approaches require large volumes of labeled data and may produce non-physical predictions.

This thesis proposes a physics-informed residual learning architecture for virtual flow metering that integrates first-principles physical modeling with data-driven machine learning. A baseline physics-based flow model is first calibrated independently for each well using routinely available pressure, temperature, and choke measurements. A global machine learning model is then trained to learn the residual errors between physics-based predictions and measured flow rates, thereby capturing unmodeled multiphase flow effects and systematic model discrepancies. During inference, physics predictions and learned residual corrections are combined through a hybrid mechanism governed by physically motivated gating rules to ensure robustness and physical consistency.

The proposed framework is evaluated using historical multi-well production data under sparse measurement conditions. Model performance is assessed using standard regression metrics and compared against physics-only and purely data-driven baseline approaches. Results demonstrate that the physics-informed residual learning framework improves flow rate estimation accuracy while maintaining interpretability and robustness, highlighting its suitability for practical deployment in industrial virtual flow metering applications.
\end{abstractpage}


% This prints the Table of Contents. Do not change this block.
\begin{ToC}
\makeatletter
    \@starttoc{toc}
 \makeatother
\end{ToC}

% This prints the List of Figures, if there is at least one figure in your document, set up according to the given guidelines.
\iffigures
    \begin{LoF}
        \makeatletter
            \@starttoc{lof}
        \makeatother
    \end{LoF}
\fi


% This prints the List of Tables, if there is at least one table in your document, set up according to the given guidelines.
\iftables
    \begin{LoT}
    \makeatletter
        \@starttoc{lot}
    \makeatother
    \end{LoT}
\fi

% This prints the List of Abbreviations, if there is at least one abbreviation in your document, set up according to the given guidelines.
\ifabbrs
    \printacronyms
\fi

% This prints the List of Appendices, if there is at least one abbreviation in your document, set up according to the given guidelines.
\ifappens
    \begin{LoAp}
    \makeatletter
        \@starttoc{apc} % Print List of Appendices
    \makeatother
    \end{LoAp}
\fi

% This marks the start of your chapters
\startContent


\chapter{Introduction}
\label{sec:intro}
The oil and gas industry is a global sector focused on the exploration, extraction, production, transportation, refining, and commercialization of hydrocarbons. As one of the most critical energy industries, it supplies fuels, feedstocks, and raw materials essential for industrial, commercial, and domestic use. The industry is characterized by complex operations spanning multiple technical, economic, and regulatory domains, requiring integration of geology, engineering, logistics, and environmental management. Hydrocarbon resources are typically categorized into crude oil, natural gas, and natural gas liquids, each of which undergoes specific extraction, processing, and distribution workflows. The industry operates across a value chain divided into three principal segments: Upstream, which involves exploration and production; Midstream, which focuses on transportation, storage, and initial processing; and Downstream, which encompasses refining, distribution, and commercialization of end-use products. 

The Upstream sector encompasses all activities related to the exploration and production of hydrocarbons. This includes geological and geophysical surveys, exploration drilling, reservoir characterization, well planning, drilling operations, completions, and production of oil, gas, and condensate. It is the segment where subsurface engineering, reservoir management, and field development strategies are applied. Central to these operations are fields and reservoirs, which define the spatial and geological context for upstream activities. A field is a geographically or geologically defined area that contains one or more hydrocarbon reservoirs. Fields represent the primary operational units for planning drilling programs, designing production facilities, and managing overall hydrocarbon recovery. They are delineated based on subsurface geology, reservoir continuity, hydrocarbon presence, and operational or administrative considerations. 

An oil well is a borehole drilled from the surface into a hydrocarbon-bearing reservoir to access and produce oil, gas, and associated fluids. Wells are designed to provide a controlled pathway for hydrocarbons to flow from the reservoir to the surface, and their design depends on reservoir properties, depth, pressure, and fluid composition. The fluid mixture flowing out of the well consists of oil, natural gas and water. This mixture of flow is called the multi-phase flow. The flow rates of oil, gas, and water reflect the performance of the well. The Petroleum Engineers need to make informed decisions regarding the production strategies based on the performance of the wells. Through accurate predictions of flow rates, petroleum Engineers can diagnose production inefficiencies and constraints, and implement solutions to optimize production and project future investments. Physical flow meters and \abbr{Virtual Flow Meter}{VFM} are used to predict flow rates in oil wells. Physical flow meters are often very expensive, require regular maintenance, and are prone to failure under harsh operating conditions. Physical flow meters use in-built sensor data and physics based mathematical equations to predict the real-time flow rates. Virtual flow meters, in contrast, can use mathematical models based on physics, models based on machine learning, or hybrid models to predict flow rates \citep{bikmukhametov2020first}. 

\section{Asset Hierarchy}
In upstream oil and gas operations, the management and organization of hydrocarbon resources are structured through a hierarchical framework known as the asset hierarchy, which provides clarity for both subsurface and surface operational planning.

\begin{itemize}
    \item \textbf{Asset}: An Asset is the highest-level operational and resource unit in upstream oil and gas management, encompassing multiple geographically or geologically defined oil and gas fields. It serves as a consolidated management entity under which exploration, development, production, and economic evaluation of the associated fields are coordinated. The asset framework enables integrated planning of drilling campaigns, reservoir development, production optimization, and infrastructure utilization across all fields it contains. By grouping multiple fields under a single asset, operators can streamline decision-making, allocate resources efficiently, and evaluate overall performance, reserves, and recovery potential at a portfolio level. In essence, the asset represents the top-level resource that aligns subsurface resources, surface facilities, and economic objectives for strategic management.
    \item \textbf{Field}: A geographically bounded development area within the asset that contains surface facilities and one or more subsurface hydrocarbon-bearing reservoirs. The field is the primary operational unit for drilling, production, and facility management.
    \item \textbf{Reservoir}: A distinct geological formation within the field that contains hydrocarbons in porous and permeable rock. Reservoirs may be stacked vertically or distributed laterally. They define the subsurface domain for fluid flow and reservoir engineering analysis.
    \item \textbf{Sector}: A subdivision of a reservoir or field created for reservoir simulation, surveillance, or operational segmentation. Sectors capture heterogeneity, fault compartments, pressure regimes, or modeling boundaries that enhance analytical accuracy.
    \item \textbf{Cluster}: A group of wells or subsurface drainage areas within a sector or field that are managed together based on operational similarity, shared production routing, or common surveillance requirements. Clusters serve as practical organizational units for production monitoring, optimization, and VFM-based diagnostics.
    \item \textbf{Wellpad}: A Wellpad is a surface facility that hosts one or more wells drilled into the subsurface to access hydrocarbon reservoirs. Wellpads provide the interface between subsurface production and surface infrastructure, including drilling rigs, flowlines, manifolds, and gathering systems. They are strategically located to optimize drilling efficiency, minimize environmental footprint, and facilitate centralized management of multiple wells. Wellpads serve as the operational unit for drilling campaigns, completions, and surface-based production monitoring, and they often group wells that belong to the same operational cluster or reservoir drainage area.
    \item \textbf{Well}: A Well is an individual borehole drilled from a wellpad into a reservoir to extract hydrocarbons or inject fluids. Each well represents a discrete production or injection point and can be monitored, managed, and optimized independently. Wells are assigned to clusters based on shared production characteristics or operational strategies, enabling efficient resource management. They may be vertical, deviated, or horizontal depending on reservoir geometry and development objectives. Wells serve as the primary conduit for connecting subsurface reservoirs to surface production systems.
    \item \textbf{Wellbore / Well String}: A Wellbore, often referred to as a well string, represents the drilled path of a well, including all installed casing, tubing, completions, and downhole equipment. A single well may contain multiple wellbore strings targeting different zones, formations, or reservoirs, allowing for separate production streams, reservoir compartment management, or enhanced recovery strategies. Wellbore design directly influences flow capacity, pressure profiles, and production efficiency, and it provides the physical framework for subsurface monitoring, artificial lift installation, and advanced reservoir management techniques.
\end{itemize}

\section{Well Types}
\begin{itemize}
    \item \textbf{Production Wells}: Production wells are the primary pathways through which hydrocarbons—oil, gas, and associated formation water—are extracted from a reservoir and transported to surface processing facilities. Their core function is to withdraw fluids from subsurface formations in a controlled manner that optimizes recovery while maintaining well integrity and reservoir performance. The performance of a production well is driven by the pressure differential between the reservoir and the wellbore. Early in a field’s life, natural energy—such as solution gas drive, water drive, gas cap expansion, or aquifer support—pushes fluids toward the wellbore. As reservoir pressure declines over time, production wells may transition to artificial lift systems such as gas lift, \abbr{Electric Submersible Pumps}{ESP}, \abbr{Progressive Cavity Pumps}{PCP}, or \abbr{Rod Pumps}{RP} to sustain flow rates and overcome the hydrostatic and frictional losses in the wellbore. The choice of artificial lift depends on fluid properties, well depth, deviation, GOR, water cut, and economic considerations.
    \item \textbf{Injection Wells}: Injection wells are specialized wellbores designed to introduce fluids—such as water, gas, or chemical mixtures—into subsurface formations for the purposes of reservoir pressure maintenance, \abbr{Enhanced Oil Recovery}{EOR}, or safe disposal of produced water. Unlike production wells, which are engineered to withdraw hydrocarbons, injection wells primarily function as input conduits, pushing fluids into the reservoir to support long-term field productivity. Their role is essential in secondary and tertiary recovery processes, as they improve sweep efficiency and maintain reservoir pressure, thereby enhancing the deliverability of adjacent production wells.
\end{itemize}

\section{Well Downhole and Surface Instrumentation}

The configuration of instrumentation in an oil well varies depending on the operational objectives and reservoir characteristics. In general, sensors are strategically installed along the wellbore and surface facilities to monitor drilling, production, and safety-related parameters. The wellbore, often referred to as the borehole, is the cylindrical hole drilled from the surface into subsurface formations during the development of oil, gas, water, or geothermal wells. Figure~\ref{fig:well-sensors} illustrates the typical placement of instrumentation across different locations of a producing well.

\subsection{Downhole Instrumentation}

\abbr{Well Down Hole}{WDH} instrumentation refers to sensors installed at or near the bottom of the well, often integrated within downhole tools or permanent monitoring systems. These sensors are designed to capture critical information about the subsurface environment and wellbore conditions. Commonly measured parameters include pressure, temperature, and fluid properties, with some advanced systems also capable of acquiring seismic or acoustic data.

The primary downhole instruments used in this study are the \abbr{Down Hole Pressure}{DHP} and \abbr{Down Hole Temperature}{DHT} sensors. These measurements provide real-time insight into reservoir behavior and well performance, enabling informed decision-making related to production optimization, reservoir management, and early detection of anomalies. Downhole data are particularly valuable for assessing reservoir drawdown, diagnosing flow restrictions, and ensuring operational safety throughout the well lifecycle.

\subsection{Wellhead Instrumentation}

\abbr{Well Head}{WH} sensors are installed at the wellhead, which represents the interface between the subsurface wellbore and surface production facilities. Wellhead instrumentation plays a crucial role in monitoring surface-operating conditions and ensuring safe and efficient production.

Typical wellhead measurements include pressure and temperature, obtained using the \abbr{Well Head Pressure}{WHP} and \abbr{Well Head Temperature}{WHT} sensors, respectively. These measurements are essential for monitoring well integrity, detecting abnormal operating conditions, and supporting production control strategies. Wellhead data also serve as key inputs for production surveillance systems and virtual flow metering applications.

\subsection{Downstream Choke Instrumentation}

The choke system is located downstream of the wellhead and is used to regulate the flow of produced fluids from the well. Choke valves control the flow rate by introducing a pressure drop, thereby enabling operators to manage reservoir drawdown, prevent sand production, and maintain well integrity.

Instrumentation associated with the choke includes the \abbr{Downstream Choke Valve}{DCR}, which represents the choke opening or valve position, and the \abbr{Downstream Choke Pressure}{DCP}, which measures the pressure downstream of the choke. Choke valves may be manually operated or automatically controlled as part of a production control system. Their placement downstream of the wellhead provides flexibility in regulating flow while maintaining safe operating conditions.

\subsection{Role of Instrumentation in Virtual Flow Metering}

All or a subset of the aforementioned sensors are used as independent variables in machine learning (ML)-based Virtual Flow Metering (VFM) models. Measurements from downhole, wellhead, and choke locations collectively capture the pressure–temperature–flow relationships governing multiphase production. The integration of these sensor signals enables data-driven models to infer oil, gas, and water flow rates in the absence of direct physical flow meters, thereby supporting real-time production monitoring and optimization.


\begin{figure}[!htb]
\uomFig{Well downhole and surface instrumentation}{fig:well-sensors}{\includegraphics[width=0.9\textwidth]{images/well-sensors.png}}{}
\end{figure}

\begin{table}[!htb]
\uomTable{Well downhole and surface instrumentation}{tab:well-sensor}{\begin{tabularx}{0.9\textwidth}{lYYYY} 
\hline
Sensor & Description & Unit \\
\hline
dhp & downhole pressure & bara \\
dht & downhole temperature & \textdegree C \\
whp & wellhead pressure & bara \\
wht & wellhead temperature & \textdegree C \\
dcp & Downstream choke pressure & bara \\
choke & Downstream choke regulation & \% \\
\hline
\end{tabularx}}{}
\end{table}

\section{Flow States}

Fluid flow within an oil and gas well occurs through a complex piping network and exhibits a wide range of physical behaviors. The flow is inherently multiphase and is influenced by well geometry, flow direction, and temporal variations in operating conditions. These characteristics collectively contribute to the complexity of flow-rate measurement and modeling in producing wells.
Fluid flow in oil and gas wells typically consists of a multiphase mixture of oil, gas, and water. The relative proportions of these phases vary spatially along the wellbore and temporally with changing reservoir and operating conditions. Multiphase interactions give rise to complex flow phenomena such as phase slip, interfacial instabilities, and regime transitions, which significantly complicate both physical measurement and data-driven estimation of flow rates.
The piping network of an oil and gas well may include vertical, deviated, inclined, and horizontal sections. As a result, fluid flow can occur in upward, downward, or mixed directions depending on the well trajectory. Changes in inclination strongly affect phase distribution, gravitational segregation, and pressure losses, thereby influencing the observed flow behavior and sensor measurements along the well.

\subsection{Steady State Flow}

Steady state flow refers to operating conditions under which the pressure, flow rates, and phase fractions of the produced fluids remain approximately constant over time. In this regime, spatial and temporal gradients of flow variables are relatively small, and the system can be reasonably assumed to be in equilibrium. Steady-state assumptions are commonly employed in analytical flow models and production surveillance due to their mathematical simplicity and reduced computational complexity. However, such assumptions may not fully capture the dynamic behavior of multiphase flow in real well operations.

\subsection{Unsteady State Flow}

Unsteady state flow is characterized by temporal variations in pressure, flow rates, and phase distribution within the wellbore. These fluctuations may arise from operational changes such as choke adjustments, variations in reservoir inflow, or transient flow phenomena including slugging and phase redistribution. In unsteady conditions, the flow variables are continuously evolving, violating steady-state assumptions and increasing the uncertainty of flow-rate measurements. Capturing unsteady-state behavior is therefore critical for accurate production monitoring and motivates the use of advanced data-driven and hybrid modeling approaches.

\subsection{Implications for Flow Measurement}

The combined effects of multiphase flow, complex well geometry, variable flow direction, and unsteady operating conditions present significant challenges for accurate flow-rate measurement in oil and gas wells. These complexities limit the reliability of conventional physical flow meters and motivate the use of virtual flow metering approaches for production monitoring \citep{bikmukhametov2020first}.

\section{Physical Flow Meters}
Physical flow meters encompass a diverse range of instruments designed to measure the rate of fluid flow across various industries and applications. These meters operate based on physics principles, employing mechanisms like pressure differentials, velocity measurement, displacement, mass calculation, or volumetric methods. Their versatility lies in their ability to cater to a wide array of fluids, from gases and liquids to multi-phase mixtures. Even in the physical flow meters, flow rates are calculated indirectly using physics based mathematical equations since it's impossible to have flow rates measured directly. Physical flow meters in general have the limitation of not being able to back-fill historical flow rates prior to the installation of the instrument. 

\subsection{Single Phase Flow Meter}
Traditional \abbr{Single Phase Flow Meter}{SPFM} systems necessitate the complete separation of the phases within the well streams before the measurement. In production metering, this separation typically occurs naturally at the exit of a conventional processing facility. These plants are designed to amalgamate various well streams at one end and deliver stabilized individual phases for transport at the other end. Typically, single-phase metering systems offer top-notch accuracy in measuring hydrocarbon production \citep{corneliussen2005handbook}. In production system a \abbr{Well Test}{WT} separator is used to physically separate the individual phases of the produced fluid so that each phase can be accurately measured and analyzed. Once the phases are separated, dedicated SPFM are employed to measure the flow rates of each individual phase of oil, gas and water. Although not as advanced as multi-phase counter parts, the SPFM are also considered expensive and its usage is extremely costly due to production deferment. This is because the well has to be disconnected from the production and needs to be connected with the separator prior to feeding the SPFM. Single phase flow measurement is often called a well test. It's possible that some wells cannot be measured from a well test because the distance from the well to the separator can be very long. Oilfields are required to perform well-tests at some point based on regulatory requirements of the jurisdictions. It is however not economically feasible to run regular well-tests as they cause production deferment.
 
\subsection{Multi Phase Flow Meter}
\abbr{Multi Phase Flow Meters}{MPFM} in the oil industry is a specialized device designed to measure the flow rates of oil, gas, and water as they are produced from a well. MPFM measure the flow rates of different phases simultaneously without separation. MPFM can provide continuous measurement of flow rates at sensor locations with decent accuracy but they are considered extremely expensive and requires regular calibration. Multi-phase flow measurement technology may be an attractive alternative since it enables measurement of unprocessed well streams very close to the well. The use of MPFM may lead to cost savings in the initial installation. However, due to increased measurement uncertainty, a cost-benefit analysis should be performed over the life cycle of the project to justify its application. It operates by determining the individual oil, gas and water phase fractions and their flow rates within a single pipeline or wellbore. MPFM also considered less accurate outside the normal operational range. SPFM are considered more accurate than MPFM since the phases are measured after separation in SPFM \citep{meribout2020multiphase}. 

\section{Virtual Flow Meters}
VFM are virtual model for predicting flow rates. Unlike the physical flow meters, VFM doesn't require deployment of dedicated hardware for flow rate measurement. Both VFM and physical flow meters measure the flow rates indirectly. Physical flow meters bring their own set of sensors whereas VFM use the existing sensors in the well. Over the past two decades, the evolution of VFM has led to the creation of diverse techniques for estimating multi-phase flow rates using field data. This progress has seen the emergence of commercial VFM systems from various companies, widely adopted by oil and gas companies globally. Physical flow meters use physics based mathematical equations to calculate the flow rates whereas VFM can use physics equations and/or ML based models for prediction. VFM, therefore can be categorized into 3 types; physics based, ML based and hybrid flow meters.

\subsection{Physics-based VFM}
Physics based VFM are also known as first principles VFM systems. Physics-based VFMs are grounded in first-principles models that describe the behavior of multiphase fluids flowing through wells and pipelines using the fundamental laws of fluid mechanics, thermodynamics, and heat transfer. These systems predict oil, gas, and water flow rates by solving mass, momentum, and energy conservation equations, typically coupled with industry-standard multiphase flow correlations or mechanistic models to compute pressure gradients, holdup, and flow regimes under varying operating conditions. Key components of physics-based VFMs include accurate \abbr{Pressure Volume Temperature}{PVT} models to represent phase behavior, inflow and vertical lift performance relationships to describe reservoir-to-surface coupling, and wellbore or pipeline geometries that define boundary conditions. By continuously ingesting real-time measurements such as pressure, temperature, choke position, and wellhead conditions, the VFM iteratively adjusts the multiphase flow solution to match the observed system behavior. This enables the model to reproduce dynamic changes in well performance—such as variations in GOR, water cut, reservoir pressure, or artificial-lift settings—without physical multiphase meters. The strength of physics-based VFMs lies in their robustness, as they rely on immutable physical laws rather than purely statistical relationships, making them highly reliable in complex or rapidly changing flow conditions where data-driven models may fail or extrapolate poorly. Following are the main stages of physics-based well modelling used in well performance analysis, production forecasting, and Virtual Flow Metering. 

\begin{itemize}
    \item \textbf{Data Acquisition and Model Initialization}:
    Physics-based well modeling begins with the collection and validation of all essential input data that define the well’s physical and thermodynamic system. This includes completion design (tubing size, deviation, perforation intervals), reservoir properties (pressure, temperature, PI, skin), and PVT data describing the phase behavior of the produced fluids. Operational parameters such as choke size, wellhead pressure, flowing temperature, and artificial-lift settings are also incorporated. The accuracy of the model depends heavily on the quality of this initial data, making data cleaning, consistency checks, and uncertainty quantification critical steps. Once validated, these inputs establish the boundary conditions for subsequent flow calculations.
    \item \textbf{Reservoir Inflow Modeling}:
    The next stage involves constructing the \abbr{Inflow Performance Relationship}{IPR}, which describes how much fluid the reservoir can deliver at different downhole pressures. Depending on reservoir type, engineers may use Vogel, Darcy, Forchheimer, or multiphase inflow models to represent inflow behavior. The IPR curve captures the interaction between reservoir pressure, permeability, fluid PVT characteristics, and near-wellbore conditions such as skin or damage. A robust IPR is essential because it determines the maximum deliverability of the well and forms the upstream boundary condition for all wellbore flow simulations.
    \item \textbf{Vertical and Multiphase Lift Modeling (VLP Curve Generation)}:
    \abbr{Vertical Lift Performance}{VLP} modeling quantifies the pressure required to lift the reservoir fluids from the perforations to the surface. This step uses mechanistic or empirical multiphase flow correlations to compute pressure gradients along the wellbore, considering gas–liquid slip, flow regime transitions, holdup, frictional losses, and hydrostatic head. Factors such as tubing geometry, inclination, heat-transfer conditions, and artificial-lift mechanisms (ESP, gas lift, PCP) strongly influence the VLP results. The resulting VLP curve provides the wellbore pressure response for any given flow rate and forms the critical link between the reservoir and surface system.    
    \item \textbf{System Matching (IPR-VLP Intersection)}:
    Once the IPR and VLP curves are generated, the model determines the stable operating point of the well by finding the intersection between the two. This intersection represents the flow rate and bottomhole pressure at which reservoir inflow and wellbore outflow are in equilibrium. System matching ensures that the well model is physically consistent and reflects actual production behavior. Engineers often calibrate this stage using well test data, pressure surveys, or downhole gauges to align the theoretical model with observed performance. Accurate matching is essential for reliable predictions in virtual metering, production optimization, and artificial lift design.
    \item \textbf{Surface Network and Choke Modeling}:
    In many wells, surface constraints significantly influence flow performance. This stage incorporates choke models, manifold pressures, flowline hydraulics, separators, and facility back-pressure effects. Choke modeling uses vendor-specific or mechanistic equations to capture the nonlinear relationship between choke opening, upstream pressure, downstream pressure, and multiphase flow rate. Incorporating the surface network allows the model to reflect operational realities, including changing separator pressures, pipeline slugging tendencies, and facility limits, thereby improving steady-state or transient flow predictions.
    \item \textbf{Calibration, Validation, and Sensitivity Analysis}:
    Physics-based well models must be calibrated against historical production data, well tests, and downhole or surface gauge measurements. Calibration typically involves adjusting uncertain parameters such as PI, skin, GOR, water cut, or tubing roughness until the model matches real observed behavior. Once calibrated, the model is validated by comparing predictions against unseen measurements to ensure predictive reliability. Sensitivity analysis is then performed to quantify how uncertainties in PVT, reservoir pressure, or equipment performance affect model outputs. This step ensures that the model remains robust, stable, and trustworthy for operational use.
\end{itemize} 



Physics based VFM dominate the industry as they've undergone significant development over half a century to comprehensively outline each facet of this approach. This extensive effort has yielded a strong grasp of mechanistic modeling within production systems, fluid properties, and optimization methods. Consequently, physics based VFM stands as a dependable method to depict overall production system behavior and especially multi-phase flow phenomena \citep{bikmukhametov2020first}. Physics based VFM are generally considered expensive due to licensing costs of the simulator and slow in prediction due to processing of high complexity mathematical models. Physics based VFM require very accurate descriptions of the well, fluids, trajectory, production choke and installation parameters for modeling the simulator based on the theories of physics. 

Physics-based VFMs can be directly compared across several important technical dimensions that determine their suitability for different production environments. One of the most important aspects is the complexity of the model, which varies significantly between vendors. Some VFMs, such as SLB’s OLGA VM and Aker’s LiveVFM, use transient multiphase flow simulations, which allow them to capture dynamic behaviors such as slugging, gas breakthrough, and changing \abbr{Gas-Oil Ratio}{GOR}. These tools are preferred for deepwater subsea wells, long tiebacks, and high-rate gas condensate production systems. Others, such as Petex PROSPER/GAP, Tieto Energy Components VFM, and BGS VMS, use steady-state mechanistic models, which are far simpler, computationally efficient, and suitable for most onshore wells or stable offshore production.

\subsection{Machine Learning VFM or Data Driven VFM}
\abbr{Machine Learning}{ML} based modeling involves analyzing system data to uncover connections between input and output variables without precise knowledge of the system's physical behavior. ML based VFM are purely data driven. This method proves advantageous in bypassing intricate physical modeling, especially for systems like multi-phase flows in pipes, where exact numerical solutions can be challenging. It relies on experimental or industrial data to grasp the system's behavior directly and attempts to learn the underlying relationships from this data. ML based VFM modeling involves collecting data such as downhole and wellhead pressures and temperatures, choke regulation levels and corresponding oil, gas, and water flow-rate measurements \citep{agwu2022modelling}. These measurements can originate from various sources in the form of well-test data or MPFM. The ML based model can serve as a backup metering system for individual wells if MPFM are installed for each wellhead. The ML based model functions as a standalone VFM system.

\subsection{Hybrid VFM}
Hybrid VFM combines the physics based and ML based models to improve the performance of the system. All 3 types of VFM rely on pressure and temperature sensor data in the well. ML based VFM and hybrid VFM also rely on physical well-test measurements for model training. More the well-test measurements available, higher the accuracy of the model within the operating conditions of the well-test measurements. They are however said to have higher degree of uncertainty outside the operating conditions of the well-test measurements. All types of VFM have the limitation of being restricted to steady state, meaning that if there are rapid transient changes in the flow, VFM will not predict them accurately. Therefore VFM cannot predict startup, shutdown and other transient scenarios. ML and hybrid VFM also cannot accurately handle advanced what-if predictive scenarios. Physics based VFM in contrast can accurately handle advanced what-if predictive scenarios.

\section{Research Problem}
\label{sec:problem}
Accurate estimation of multiphase flow rates in producing oil and gas wells is a fundamental requirement for effective production monitoring, reservoir management, and operational decision-making. Traditionally, this task is performed using MPFMs, which provide direct measurements of oil, gas, and water flow rates. However, MPFMs are capital-intensive, require regular calibration and maintenance, and are often deployed at limited locations due to cost and operational constraints. As a result, continuous and comprehensive flow rate measurements are rarely available across all wells and operating conditions.

To address these limitations, VFM techniques have been developed as software-based alternatives to physical flow meters. The conventional VFM approaches rely on {first-principles physics-based models, which use pressure, temperature, choke settings, and fluid properties to estimate flow rates. While these models are grounded in physical laws and offer interpretability, their accuracy is often limited by simplifying assumptions, uncertainty in model parameters, unmodeled multiphase flow phenomena, and changing well and reservoir conditions over time. Consequently, purely physics-based VFMs may exhibit systematic bias or degraded performance when applied outside their calibration envelope.

Recent advances in data-driven and machine learning techniques have demonstrated strong predictive capabilities for complex nonlinear systems, including multiphase flow behavior. However, purely data-driven models typically require large volumes of high-quality labeled data, which are seldom available in production environments due to sparse MPFM measurements and irregular well testing schedules. Furthermore, black-box ML models often lack physical interpretability and may generate non-physical predictions when extrapolated beyond the training data, limiting their acceptance in safety-critical industrial applications.

This situation highlights a fundamental challenge in Virtual Flow Metering: physics-based models provide interpretability and physical consistency but suffer from model inaccuracies, while data-driven models offer flexibility and expressive power but are constrained by limited data availability and a lack of physical grounding. Neither approach alone provides a fully satisfactory solution under realistic industrial conditions characterized by sparse supervision, noisy measurements, and evolving operating regimes.

A promising approach to overcome these limitations is the development of hybrid or physics-informed learning architectures, in which physical models and machine learning are combined in a complementary manner. In particular, \textit{residual learning} offers a structured framework whereby a physics-based model produces a baseline estimate of flow rates, and a data-driven component learns to model the residual errors arising from unmodeled dynamics, parameter uncertainty, and measurement noise. This formulation preserves physical interpretability while enabling data-driven correction of systematic model deficiencies.

Despite increasing interest in hybrid modeling approaches, there remains a lack of systematic investigation into physics-informed residual learning architectures for Virtual Flow Metering under sparse and irregular measurement conditions, particularly in multi-well production systems. 
Key open questions include how residual models can be trained effectively with limited ground-truth flow measurements, how physics-generated states can be leveraged as informative context without introducing data leakage, and how such architectures compare against traditional physics-only and purely data-driven methods in terms of accuracy, robustness, and generalization.
Therefore, the research problem addressed in this thesis can be stated as follows:

\begin{quote}
\textit{How can a physics-informed residual learning architecture be designed and evaluated to improve the accuracy and robustness of virtual flow metering in oil and gas wells under sparse and irregular flow rate measurements?}
\end{quote}

This research aims to bridge the gap between physics-based modeling and data-driven learning by proposing and validating a hybrid framework that exploits the strengths of both paradigms while adhering to the practical constraints of industrial oil and gas production systems.

\section{Research Objectives}
\label{sec:objectives}

The primary objective of this research is to develop and evaluate a physics-informed residual learning framework for virtual flow metering in oil and gas production wells under conditions of sparse and irregular flow rate measurements. To achieve this overarching goal, the following specific research objectives are defined:

\begin{enumerate}
    \item \textbf{To review and analyze existing virtual flow metering approaches}, including conventional physics-based models, purely data-driven methods, and hybrid modeling techniques, with particular emphasis on their strengths and limitations in industrial production environments.

    \item \textbf{To formulate a baseline physics-based virtual flow metering model} that estimates multiphase flow rates using routinely available well measurements such as pressures, temperatures, and choke settings, and to assess its performance and limitations under varying operating conditions.

    \item \textbf{To design a physics-informed residual learning architecture} in which a data-driven model is trained to learn and correct the residual errors of the baseline physics-based flow model, thereby improving prediction accuracy while preserving physical interpretability.

    \item \textbf{To develop a data preprocessing and feature engineering strategy} that enables effective training of the residual learning model under sparse and irregular ground-truth flow measurements, including the integration of physics-generated state variables as informative contextual inputs.

    \item \textbf{To develop a computationally efficient prediction framework} that enables scalable virtual flow metering across large well populations, recognizing that industrial oil and gas assets may consist of thousands of wells distributed across multiple fields and production systems.

    \item \textbf{To implement the proposed hybrid modeling architecture} using real-world multi-well production data and to ensure that the training, validation, and testing procedures avoid data leakage and reflect realistic operational constraints.

    \item \textbf{To evaluate the performance of the proposed physics-informed residual learning approach} using appropriate regression metrics, and to compare its predictive accuracy against physics-only and purely data-driven baseline models.

    \item \textbf{To analyze the robustness and generalization capability of the proposed framework} across multiple wells and operating regimes, with particular attention to its behavior under limited training data and changing system conditions.

    \item \textbf{To assess the practical applicability of the proposed approach} for industrial virtual flow metering by examining its interpretability, computational efficiency, and potential integration into existing production monitoring workflows.
\end{enumerate}


\section{Scope and Limitations}

This research focuses on the development and evaluation of a physics-informed residual learning architecture for virtual flow metering in oil and gas production wells. The scope of the study and its inherent limitations are outlined below to clearly define the boundaries of the work and to contextualize the results.

\subsection{Scope of the Research}

\begin{enumerate}
    \item The study is confined to \textit{producing oil and gas wells} equipped with standard surface and downhole instrumentation, including pressure, temperature, and choke measurements, which are routinely available in production operations.

    \item The proposed virtual flow metering framework integrates a \textit{first-principles physics-based model} with a \textit{data-driven residual learning component} to estimate multiphase flow rates, with particular emphasis on oil, gas, and water production rates.

    \item The research considers \textit{sparse and irregular ground-truth flow measurements}, such as those obtained from MPFMs or periodic well tests, reflecting realistic industrial data availability constraints.

    \item The evaluation of the proposed approach is conducted using \textit{historical production data from multiple wells}, enabling assessment of the model’s performance, robustness, and generalization capability across different wells and operating conditions.

    \item Model performance is assessed using standard regression metrics, including accuracy and error-based measures, and is compared against baseline physics-only and data-driven modeling approaches.

    \item The study emphasizes \textit{software-based virtual flow metering} and does not consider hardware design, sensor deployment strategies, or real-time control system implementation.

\end{enumerate}

\subsection{Limitations}

\begin{enumerate}
    \item The accuracy of the proposed framework is inherently dependent on the quality, availability, and representativeness of the input data, including sensor measurements and ground-truth flow rate observations.

    \item The baseline physics-based model employed in this study involves simplifying assumptions regarding multiphase flow behavior, fluid properties, and well conditions, which may not fully capture complex transient or highly nonlinear phenomena.

    \item The data-driven residual learning component is trained on historical data and may exhibit reduced performance when applied to operating regimes or well conditions that are significantly different from those observed during training.

    \item The proposed framework is evaluated using offline historical datasets and does not explicitly address real-time deployment challenges such as computational latency, online model updating, or integration with supervisory control and data acquisition (SCADA) systems.

    \item The study does not explicitly quantify uncertainty in flow rate predictions, such as confidence intervals or probabilistic uncertainty bounds, although this represents a potential direction for future research.

    \item The findings of this research are specific to the datasets and wells considered and may require recalibration or retraining before application to different reservoirs, fields, or asset types.

\end{enumerate}


\chapter{Literature Review}
\label{sec:literature}

\section{Introduction}

The accurate estimation of oil, gas, and water production rates is a fundamental requirement in petroleum production engineering, underpinning reservoir management, production optimization, flow assurance, and economic decision-making. Traditionally, phase flow rates are measured using physical multiphase flow meters (MPFMs). While MPFMs provide direct measurements, their widespread deployment is limited by high capital cost, maintenance requirements, calibration complexity, and installation constraints, particularly in offshore, subsea, and marginal field developments \citep{corneliussen2005handbook,meribout2020multiphase,hansen2019multi}.

As a cost-effective alternative, Virtual Flow Metering (VFM) has emerged as a key enabling technology. VFM systems estimate phase flow rates indirectly using routinely available surface and downhole measurements such as pressures, temperatures, and choke settings. Over the past two decades, VFM methodologies have evolved from purely physics-based models to data-driven machine learning approaches and, more recently, to hybrid physics--data architectures \citep{bikmukhametov2020first}. This chapter reviews these developments, critically examines their strengths and limitations, and establishes the motivation for a physics-informed hybrid residual learning architecture for virtual flow metering.

\section{Physics-Based Virtual Flow Metering}

\subsection{Foundations of Petroleum Production Modeling}

Physics-based VFM approaches rely on first-principles models derived from reservoir engineering, fluid mechanics, and thermodynamics. Classical petroleum engineering formulations describe the relationship between reservoir pressure, flowing bottom-hole pressure, and production rate using inflow performance relationships (IPRs) \citep{dake1978developments}. These formulations are widely used in nodal analysis to predict well performance under steady-state conditions.

IPR models typically assume homogeneous reservoir properties, steady-state flow, and known fluid characteristics. While these assumptions are reasonable during early production or controlled testing, they are frequently violated during long-term field operation, leading to systematic prediction errors.

\subsection{Choke Flow and Surface Constraints}

Surface choke flow models form a critical component of physics-based VFMs by linking upstream pressure, downstream pressure, and choke opening to flow rates. Numerous empirical and semi-empirical correlations have been developed to model multiphase flow through chokes \citep{agwu2022modelling,alrumah2019new}. These correlations are often calibrated for specific flow regimes and fluid compositions.

Although choke models provide physically interpretable relationships, their accuracy is highly sensitive to operating conditions. Flow regime transitions, changing gas–liquid ratios, and water breakthrough can significantly degrade predictive performance. Consequently, choke correlations often require frequent recalibration, limiting their robustness for long-term monitoring.

\subsection{Strengths and Limitations of Physics-Based VFMs}

Physics-based VFMs offer several advantages, including interpretability, compliance with physical laws, and the ability to extrapolate beyond observed data ranges. However, they suffer from key limitations. Uncertain reservoir parameters, incomplete representation of multiphase flow physics, and measurement noise can lead to biased predictions \citep{bikmukhametov2020first}. Moreover, mechanistic models struggle to capture complex nonlinear interactions between operational variables, motivating the exploration of data-driven alternatives.

\section{Data-Driven Virtual Flow Metering}

\subsection{Early Machine Learning Approaches}

The application of machine learning to VFM began with artificial neural networks (ANNs) and regression-based soft sensors \citep{al2017development,al2017radial}. These models learned direct mappings between sensor measurements and phase flow rates, often outperforming physics-based models under stable operating conditions.

However, early ANN-based VFMs exhibited limited generalization capability and were prone to overfitting, particularly when training datasets were small or unevenly distributed across operating regimes.

\subsection{Advanced Data-Driven Models}

Subsequent research explored ensemble learning, adaptive optimization techniques, and Bayesian neural networks to improve robustness and uncertainty quantification \citep{al2018virtual,grimstad2021bayesian}. Ensemble methods reduced variance by combining multiple learners, while Bayesian approaches provided probabilistic estimates of prediction uncertainty.
\citep{grimstad2021bayesian} develops a ML based VFM, implemented on \abbr{Bayesian Neural Network}{BNN}. BNN provide a probabilistic distribution over weights and predictions unlike traditional neural networks, which produce point estimates for weights and predictions. They have trained the model on a large and heterogeneous data-set, consisting of 60 wells across five different oil and gas assets. The predictive performance is analyzed on historical and future test data, where an average error of 4\%–6\% and 8\%–13\% is achieved for the 50\% best performing models, respectively. 

More recently, deep learning architectures have been applied to VFM, demonstrating strong predictive performance in data-rich environments \citep{song2022intelligent,mercante2022virtual}. Despite these advances, purely data-driven VFMs remain fundamentally dependent on large, high-quality datasets and often lack physical interpretability.

\citep{muchsin2023virtual} proposed a VFM model using a time series-time delay artificial neural network technique in predicting the multi-phase flow rate accurately with the best average discrepancy of Qgas (6.0\%), Qoil (-16.4\%), and Qwater (-2.4\%) compared with actual measurement by MPFM. The training of the VFM model, which takes less than 7 minutes and has acceptable values for MSE, R, and MAPE, demonstrates that this method is reliable enough to be used in the oil and gas industries, which often require conducting dozens of individual well tests in their daily activities. The utilization of data measurement of existing well orifice meter combined with data measurement from choke valve and well head as parameter data input of the network has proven effective to improve the accuracy of the VFM model. In addition, the greater the number of data used in training phase determines the accuracy performance of the VFM model. However, concerning that this study only uses limited well reference data, it is recommended to implement this VFM model on other wells in different fields to validate its performance accuracy during well testing.

\citep{al2018virtual} introduces a VFM system employing ensemble learning tailored for fields with limited data from a shared metering setup. The method creates diverse neural network learners by manipulating training data, neural network structure, and learning approach. Using \abbr{Adaptive Simulated Annealing}{ASN} optimization, it selects the best subset of learners and an optimal combining strategy. Assessment using real well test data shows exceptional performance, with average errors of 4.7\% and 2.4\% for liquid and gas flow rates respectively. The accuracy of their VFM was further confirmed through cumulative deviation analysis, where predictions fall within a maximum deviation of ±15\%. Comparisons with standard bagging and stacking techniques demonstrate significant enhancements in both accuracy and ensemble size. The proposed VFM system offers ease in development and maintenance compared to traditional model-driven VFM, requiring only well test samples for model tuning. It is anticipated that this developed VFM can complement physical meters, improve data consistency, aid in reservoir management and flow assurance, ultimately resulting in more efficient oil recovery and reduced operational and maintenance costs.

\citep{al2017radial} introduces a radial basis function network designed to create a virtual flow meter (VFM) specifically for estimating gas flow rates within multi-phase production lines. Validating the model with real well test data demonstrates its outstanding performance and ability to generalize effectively. Furthermore, the paper delves into the importance of downhole and choke valve measurements in ensuring precise predictions. This proposed VFM model presents a potentially appealing and cost-efficient solution for real-time production monitoring needs while simultaneously reducing operational and maintenance expenses.

In a separate work also by \citep{al2017development} proposes a method for estimating phase flow rates in oil and gas production wells using readily available measurements. By overcoming the limitations of traditional metering facilities, their system offers a cost-effective way to monitor production in real time. It not only cuts operational and maintenance expenses but also serves as a reliable backup to multi-phase flow meters. The technique involves creating a soft-sensor through a feed-forward neural network. To ensure accuracy without excessive complexity, the system uses K-fold cross-validation and an early stopping technique to regulate generalization and network intricacy. Validation using real well test data demonstrates the sensor's effectiveness, evaluated through cumulative deviation and flow plots. Their results indicate promising performance, with an average error of approximately 4\% and less than 10\% deviation for 90\% of the samples.

\citep{song2022intelligent} employs \abbr{Back Propogation Neural Network}{BPNN}, LSTM network, and \abbr{Random Forest}{RF} algorithm to develop an intelligent ML based model for virtual flow meters in oil and gas development. Their data-set is constructed using actual data from two oil wells in an offshore oil field in the South China Sea. Among the three models, the LSTM model has demonstrated the highest accuracy, with a Mean Absolute Error (MAE) of 3.9\%. LSTM has also demonstrated the highest stability and requires a moderate amount of data volume. BP network on the other hand have exhibited the lowest accuracy, with a MAE of 12.1\%, as well as the lowest stability. BP however has shown the smallest data volume requirement. The Random Forest model has shown moderate accuracy, high stability, but has required the highest data volume.

\subsection{Time-Series and Attention-Based Models}

The increasing availability of time-stamped production data has motivated the use of time-series models in VFM. Recurrent neural networks and long short-term memory (LSTM) networks have been used to capture temporal dependencies in production behavior \citep{andrianov2018machine}.
This is because unlike feed-forward neural network, which process input data in a one-way, LSTM-RNN are designed to process sequential data, such as time series. \citep{andrianov2018machine} has achieved the best accuracy when the lengths of the input and output sequences to LSTM are equal. 

\subsection{Limitations of Purely Data-Driven VFMs}

Despite strong predictive performance, purely data-driven VFMs exhibit fundamental limitations. They lack explicit physical grounding, struggle with extrapolation beyond the training domain, and may violate physical constraints such as non-negativity or phase consistency \citep{bikmukhametov2020first}. The black-box nature of deep learning models also reduces trust and hinders adoption in industrial decision-making contexts.

\section{Hybrid Physics - Data Virtual Flow Metering}

\subsection{Motivation for Hybrid Modeling}

Hybrid modeling approaches seek to combine the strengths of physics-based and data-driven models. In VFM applications, physics-based models capture dominant flow behavior, while machine learning models excel at learning unmodeled nonlinearities and systematic discrepancies \citep{hotvedt2020developing}.
The hybrid VFM proposed by \citep{ishak2022virtual} uses ensemble learning to develop the data driven model by incorporating multiple ML models. The data driven model is then combined with the physics model using a combiner. They have achieved a 50\% improvement in performance using the combiner compared to their stand-alone performance of physics based and ML based VFM. 

\citep{nemoto2023cloud} introduces a cloud-based VFM that combines physics and data techniques to accurately estimate water production per well within a gas field. By integrating physics-based models tailored for high gas volume fraction gas-liquid flows in the wellbore and employing a data-driven approach to fine-tune these models with actual well test data, this hybrid method ensures precise real-time estimations. Its adaptability to evolving well performance and increased water production creates a self-calibrating solution, ensuring ongoing accuracy and relevance despite changing production and well conditions. Their VFM system demonstrates substantial alignment with well test data under steady-state conditions, affirming its reliability. Operating remotely through a cloud-based Cognite DataOps platform, this system conducts calculations and stores results, facilitating continual access to live sensor data for other applications or visualization via a web interface. Utilizing existing sensors within the wells, the VFM system offers cost efficiency by minimizing both initial investment and operational expenses in comparison to installing multi-phase flow meters or separators.

Hybird VFM models have become increasingly popular since 2018 and increase usage of neural networks seen for developing both physics and ML based models.

\subsection{Residual Learning Frameworks}

Residual learning has emerged as a particularly effective hybrid strategy. In this framework, a physics-based model provides baseline predictions, and a machine learning model is trained to learn residual errors between predictions and measurements \citep{bikmukhametov2020combining}. This separation improves data efficiency, stabilizes training, and preserves physical interpretability.

Residual hybrid VFMs have demonstrated improved accuracy and robustness compared to both purely physics-based and purely data-driven approaches \citep{ishak2022virtual,gryzlov2023combining}.

\subsection{Industrial Applications}

Hybrid VFMs have been successfully deployed in industrial environments, including cloud-based real-time monitoring systems for offshore fields \citep{nemoto2023cloud}. These studies highlight the scalability and operational relevance of hybrid architectures, particularly in settings with sparse instrumentation and limited well-test data.

\section{Physics-Informed Machine Learning}

\subsection{Physics-Informed Neural Networks}

Physics-Informed Neural Networks (PINNs) incorporate governing equations directly into the training process and have been explored for flow metering applications \citep{franklin2022physics}. While PINNs offer a principled approach to enforcing physical laws, their application to multiphase well flow remains challenging due to incomplete or uncertain governing equations.
\citep{franklin2022physics} proposes a hybrid VFM combining \abbr{Physics Informed Neural Networks}{PINN} based physics model and LSTM-RNN based ML model. Their resulting hybrid model is capable of predicting the average flow rate some time-steps ahead using the measurements available in the oil well information system. Unlike the ML based LSTM-RNN model proposed by \citep{andrianov2018machine}, the system proposed by \citep{franklin2022physics} is hybrid model using neural networks for physics based model as well as the ML based model. 

\subsection{Soft Physics Constraints and Closure Laws}

An alternative to strict physics enforcement is the use of soft physical constraints embedded in model architecture. Examples include bounded logistic water-cut closures and explicit non-negativity constraints on phase rates \citep{bikmukhametov2020first}. These approaches improve physical realism while retaining flexibility and numerical stability.

\section{Rationale for a Physics-Informed Hybrid Residual Architecture}

The adoption of a physics-informed hybrid residual architecture in this work is motivated by several well-established considerations in the virtual flow metering and data-driven modeling literature:

\begin{itemize}
    \item \textbf{Limited availability and sparsity of well test data:}  
    In practical production environments, well test measurements are acquired infrequently due to operational constraints, cost, and production priorities. 
    Purely data-driven models typically require large, dense, and regularly sampled datasets to generalize reliably, which is rarely achievable in field conditions. 
    Physics-based models, by contrast, can operate under sparse data regimes by embedding prior physical knowledge, making them a natural foundation for hybrid approaches.

    \item \textbf{Strong prior knowledge of multiphase flow behavior:}  
    The governing relationships between pressure, temperature, fluid properties, and phase flow rates in production systems are well understood and extensively documented in the petroleum engineering literature. 
    Ignoring this prior knowledge and learning flow behavior purely from data is inefficient and may lead to physically implausible predictions. 
    A physics-informed architecture ensures that baseline predictions remain consistent with known flow physics.

    \item \textbf{Systematic bias and model mismatch in physics-based models:}  
    While physics-based virtual flow metering models capture dominant flow mechanisms, they often rely on simplifying assumptions, empirical correlations, and uncertain parameters. 
    As a result, systematic residual errors arise due to unmodeled effects such as changing flow regimes, sensor drift, fluid property variations, and facility interactions. 
    Learning residual corrections allows these structured biases to be addressed without discarding the physical model.

    \item \textbf{Improved generalization across wells and operating conditions:}  
    Residual learning focuses the machine learning component on modeling deviations from physics rather than the full input--output mapping. 
    This reduces model complexity and improves generalization, particularly when transferring predictive capability across wells with heterogeneous characteristics. 
    Such decomposition has been shown to enhance robustness under spatial and temporal distribution shifts.

    \item \textbf{Reduced risk of overfitting and improved data efficiency:}  
    By constraining the learning task to residual errors, the hybrid architecture significantly reduces the hypothesis space explored by the machine learning model. 
    This improves data efficiency and mitigates overfitting, which is critical when training on small multi-well datasets typical of well test applications.

    \item \textbf{Interpretability and diagnostic value:}  
    Separating physics-based predictions from learned residual corrections enhances model interpretability. 
    The physics component provides transparent baseline behavior, while the residual component highlights conditions under which the physical assumptions break down. 
    This separation supports diagnostic analysis and builds trust for deployment in production engineering workflows.

    \item \textbf{Alignment with operational calibration practices:}  
    In field operations, physics-based models are routinely calibrated using limited well-specific data, while global correlations and assumptions are reused across assets. 
    The proposed hybrid architecture mirrors this practice by allowing limited per-well calibration of physical parameters while maintaining a globally trained residual model, thereby aligning model design with real-world engineering workflows.

    \item \textbf{Robustness under non-stationary operating conditions:}  
    Production systems evolve over time due to changes in reservoir conditions, choke settings, artificial lift performance, and facility constraints. 
    Hybrid residual models can adapt to such non-stationarities by updating residual behavior while retaining physically consistent baseline predictions, offering greater long-term stability than purely data-driven approaches.
\end{itemize}


\newcolumntype{Z}{>{\raggedright\arraybackslash}p{2.5cm}}
\begin{longtable}{p{2cm} Z Z Z Z}
\caption{Comparative evaluation of flow metering and virtual flow metering approaches}
\label{tab:flow_metering_comparison} \\

\hline
\textbf{Criterion} 
& \textbf{MPFM} 
& \textbf{Physics-Based VFM} 
& \textbf{ML-Based VFM} 
& \textbf{Hybrid Physics-ML VFM} \\
\hline
\endfirsthead

\multicolumn{5}{c}{{\tablename\ \thetable{} -- continued}} \\
\hline
\textbf{Criterion} 
& \textbf{MPFM} 
& \textbf{Physics-Based VFM} 
& \textbf{ML-Based VFM} 
& \textbf{Hybrid Physics-ML VFM} \\
\hline
\endhead

\hline \multicolumn{5}{r}{{continued on next page}} \\
\endfoot

\hline
\endlastfoot

Measurement principle 
& Direct physical measurement using nuclear, differential pressure, impedance, or venturi-based sensors 
& Governing mass, momentum, and energy balance equations with empirical closure relations 
& Statistical or machine-learning-based mapping between sensor inputs and flow rates 
& Physics-guided data-driven inference combining physical laws and machine learning \\

Accuracy under ideal conditions 
& High when calibrated and operated within design envelope 
& Moderate to high, depending on model fidelity and calibration 
& High within training data domain 
& High across wide operating ranges due to physics-guided learning \\

Performance under changing flow regimes 
& Degrades under slugging, emulsions, and severe regime transitions 
& Sensitive to regime assumptions and closure law validity 
& Poor extrapolation beyond trained regimes 
& Robust due to physical consistency and adaptive correction \\

Data requirements 
& Minimal (sensor signals only) 
& Low to moderate (well geometry, PVT properties, boundary conditions) 
& High (large, labeled datasets required) 
& Moderate (limited labeled data supplemented by physics constraints) \\

Dependence on well tests 
& Required periodically for recalibration 
& Strong dependence for parameter estimation 
& Strong dependence for supervised learning 
& Reduced dependence due to physics-informed training \\

Generalization capability 
& Limited to hardware operating envelope 
& Limited across wells due to well-specific parameters 
& Limited; prone to overfitting 
& Improved cross-well and cross-condition generalization \\

Physical interpretability 
& High (direct physical measurements) 
& High (explicit physical meaning of parameters) 
& Low (black-box behavior) 
& High to moderate with interpretable residual structure \\

Computational cost 
& Low during operation; high installation and maintenance cost 
& Moderate due to nonlinear solvers 
& Low inference cost; high training cost 
& Moderate; suitable for near real-time deployment \\

Installation and operational cost 
& Very high (CAPEX and OPEX) 
& Low (software-based) 
& Low (software-based) 
& Low (software-based) \\

Sensitivity to sensor noise 
& Moderate; hardware filtering required 
& High sensitivity due to noise propagation through equations 
& High unless explicitly regularized 
& Reduced through physics-based regularization \\

Scalability across fields 
& Poor due to hardware deployment constraints 
& Moderate with per-well calibration 
& High if sufficient data exists 
& High with minimal re-calibration \\

Long-term maintainability 
& Challenging due to sensor drift and failures 
& Moderate; requires periodic recalibration 
& Challenging due to model drift and retraining 
& Improved stability via physics constraints \\

\end{longtable}


\begin{longtable}[!htb]{p{0.2\textwidth}p{0.1\textwidth}p{0.2\textwidth}p{0.35\textwidth}}
\uomTable{Literature summary}{tab:literature-summary}{}{}
\cr 
\hline
Research & VFM Type & ML Model & Evaluation Metrics \\
\hline
\endfirsthead
\multicolumn{2}{l}%
{{\tablename\ \thetable{} -- continued}} \\
\hline
Research & VFM Type & ML Model & Evaluation Metrics \\
\hline
\endhead % Header for subsequent pages
\cite{al2017radial} & ML & ANN & Oil) \(R^{2} = 0.965, RMSE = 1.24899, MAPE = 4.22\) Gas) \(R2 = 0.954, RMSE = 1.35277, MAPE = 2.27\) \\
\cite{al2017development} & ML & RBF & \(R^{2} = 0.93978, RMSE = 1.334, MAPE = 6.16\) \\
\cite{andrianov2018machine} & ML & LSTM, RNN & Not reported \\
\cite{al2018virtual} & ML & Ensemble NN and ASN & \(ANN: RMSE = 0.0585; STDEV = 0.0046; MAPE = 4.7\) \(ASA: RMSE = 0.0442, STDEV = 0.0036, MAPE = 2.35\) \\
\cite{dutta2018modeling} & ML & ANN FPA & \(RMSPE = 0.75, ARPE = 99.25\) \\
\cite{hansen2019multi} & ML & ANN & $\text{\(a) R = 0.993, AAPE = 8.39\)}$ $\text{\(b) R = 0.995, AAPE = 6.36\)}$ \\
\cite{alrumah2019new} & ML & ANN, LSSVM, SIMPLEX & \(ANN: R2 = 0.9292, RMSE = 863.98, AARPE = 22.06\) \(LSSVM: R^{2} = 0.9477, RMSE = 719.6, AARPE = 21.5\) \(SIMPLEX: R^{2} = 0.885, RMSE = 1067, AARPE = 26.8\) \\
\cite{bikmukhametov2020combining} & ML & MLP, LSTM, GB & \(Oil:\) \(MLP: RMSE = 0.0458; LSTM: RMSE = 0.0476\) \(GB: RMSE = 0.0463\) \(Gas:\) \(MLP: RMSE = 0.0328, LSTM: RMSE = 0.278\) \(GB: RMSE = 0.0367\) \\
\cite{hotvedt2020developing} & Hybrid & Not reported & \(RMSE = 15, MAE = 8\) \\
\cite{marfo2021predicting} & ML & ANN & \(R = 0.9966, MAPE = 3.18\) \\
\cite{grimstad2021bayesian} & ML & BNN & \\
\cite{ishak2022virtual} & Hybrid & Ensemble model & \\
\cite{song2022intelligent} & ML & BPNN, LSTM, Random Forest & \\
\cite{franklin2022physics} & Hybrid & PINN, LSTM, RNN & \(10^{-4} \leq MSE \leq 10^{-3}\) \\
\cite{muchsin2023virtual} & ML & TSTD-ANN & \(MSE\leq 1500, R^{2} \geq 0.95, MAPE \leq 65\) \\
\cite{nemoto2023cloud} & Hybrid & Linear Regression & M1) \(MRPE = 12.64, MAPE = 20.61, R^{2} = 0.77\) M2) \(MRPE = 0.43, MAPE = 17.43, R^{2} = 0.80\) M3) \(MRPE = -0.23, MAPE = 20.10, R^{2} = 0.64\) \\
\hline
\end{longtable}


\chapter{Methodology}
\label{chap:methodology}

\section {Overview of the Proposed Architecture}

This thesis proposes a physics-informed residual learning architecture for virtual flow metering that combines first-principles physical modeling with data-driven machine learning to estimate multiphase flow rates in oil and gas production wells. The framework is designed to exploit the complementary strengths of physics-based models and data-driven approaches while addressing the practical constraints of sparse and irregular flow rate measurements typically encountered in industrial environments.

The overall architecture consists of three main components:

\begin{enumerate}
    \item \textbf{A baseline physics-based flow model}, calibrated independently for each well, which provides first-order estimates of oil, gas, and water flow rates based on physically interpretable relationships and routinely available well measurements.
    \item \textbf{A global data-driven residual learning model}, trained across multiple wells to learn the systematic residual errors between physics-based predictions and measured flow rates, thereby capturing unmodeled dynamics and nonlinear effects.
    \item \textbf{A hybrid inference mechanism}, which combines the physics-based predictions with the learned residual corrections under physically motivated gating rules to ensure robustness, physical plausibility, and stable performance under varying operating conditions.
\end{enumerate}

A schematic representation of the architecture is conceptually illustrated in Figure~\ref{fig:component_architecture}.

\begin{figure}[!htb]
\uomFig{Component-level architecture of the proposed hybrid physics-informed residual learning framework}{fig:component_architecture}{\includegraphics[width=1.0\textwidth]{images/component_architecture.png}}{}
\end{figure}

\subsection{Baseline Physics-Based Inflow Modeling}

At the core of the proposed framework lies a physics-based virtual flow metering model that provides first-order estimates of oil, water, and gas flow rates. This model is formulated using simplified physical relationships that capture the dominant dependencies between flow rates and routinely available well measurements, including downhole pressure, choke opening, downstream pressure, and temperature measurements.
The physics model incorporates:

\begin{itemize}
    \item A pressure-driven liquid inflow relationship governed by an effective reservoir pressure
    \item A water-cut formulation expressed through a logistic function of pressure, temperature, and choke-related features
    \item A gas flow component modeled as a pressure-drop-driven relationship modulated by choke effectiveness
\end{itemize}

Model parameters are calibrated on a per-well basis using nonlinear least-squares optimization against available multiphase flow meter measurements. This calibration step ensures that the physics model remains interpretable, physically consistent, and tailored to individual well characteristics while avoiding excessive model complexity.

\subsection{Residual Learning via Machine Learning}

While the physics-based model captures the dominant flow mechanisms, it is inherently limited by simplifying assumptions, parameter uncertainty, and unmodeled multiphase flow effects. To address these deficiencies, a data-driven residual learning component is introduced.

Rather than predicting flow rates directly, the machine learning model is trained to learn the residual error between measured flow rates and physics-based predictions. Residuals are formulated in logarithmic space to stabilize variance and to ensure physically meaningful corrections, particularly at low flow rates.
The residual model:

\begin{itemize}
    \item Operates globally across all wells to leverage shared flow behavior
    \item Uses both measured operational variables and physics model predictions as input features
    \item Implemented using a gradient-boosted ensemble regressor capable of modeling nonlinear interactions while maintaining robustness under limited data
\end{itemize}

Lagged pressure features are incorporated to provide limited temporal context without imposing strong sequential assumptions, enabling the model to capture delayed system responses while remaining compatible with sparse measurement regimes.

\subsection{Hybrid Inference and Physics-Guided Gating}

During inference, physics-based predictions and machine-learned residuals are combined to produce final flow rate estimates. Importantly, residual corrections are not applied indiscriminately. Instead, physically motivated gating mechanisms are employed to ensure that machine learning corrections are only activated under conditions where the physics model is expected to be unreliable.
For example:

\begin{itemize}
    \item Oil rate corrections are selectively applied under high water-cut or large residual conditions
    \item Water rate corrections are activated only when water production exceeds a minimum threshold
    \item Gas rate predictions are consistently corrected due to higher uncertainty in gas modeling under varying choke and pressure conditions
\end{itemize}

These gating strategies preserve physical plausibility, prevent non-physical predictions, and enhance model robustness when extrapolating beyond the calibration range.

\subsection{Multi-Well Structure and Model Deployment}

The proposed framework explicitly supports multi-well deployment. Physics models are calibrated independently for each well, while the residual learning component is trained globally across all wells to exploit shared multiphase flow behavior. This structure enables transfer of learned residual patterns between wells while retaining well-specific physical characteristics.

The framework supports both sparse prediction at measured timestamps and dense flow rate reconstruction for periods without direct measurements. Additionally, model persistence mechanisms are implemented to enable saving, loading, and reuse of calibrated physics models and trained machine learning components.

\subsection{Key Advantages of the Proposed Architecture}
The proposed physics-informed residual learning architecture offers several key advantages:

\begin{itemize}
    \item Improved accuracy over physics-only virtual flow metering by compensating for systematic model errors
    \item Improved robustness and physical consistency compared to purely data-driven models
    \item Suitability for sparse and irregular measurement regimes
    \item Interpretability through explicit separation of physics-based predictions and data-driven corrections
    \item Scalability to multi-well production systems
\end{itemize}

Together, these characteristics make the proposed framework a practical and scientifically grounded solution for virtual flow metering in real-world oil and gas production environments.


\section{Dataset Description and Preprocessing}
\label{sec:data_preprocessing}
The dataset used in this research consists of multi-source production and operational measurements collected from a set of producing oil wells. The data integrate well test measurements, MPFM data, and permanently installed surface and downhole sensor readings. 
This study employs the \textbf{same dataset previously utilized by \citep{nemoto2023cloud}}, thereby ensuring consistency with established literature while extending the analysis through a hybrid physics--machine learning modeling architecture.
A defining characteristic of the dataset is that \textbf{well test data are sparse and irregular in time}. Well tests are conducted at non-uniform intervals determined by operational, logistical, and economic considerations rather than by systematic sampling schemes. 
Consequently, the temporal spacing between successive well tests may range from days to several weeks or months, resulting in an unevenly sampled supervisory signal for flow rate estimation.
In contrast, auxiliary measurements such as wellhead pressure (WHP), downhole pressure (DHP), temperatures, choke settings, and downstream pressures are available at higher temporal resolution. 
Nevertheless, these signals may still exhibit missing values, sensor dropouts, or extended periods of constant readings associated with shut-ins or instrumentation limitations.

The dependent variables considered in this study are the phase flow rates—oil rate ($q_o$), water rate ($q_w$), and gas rate ($q_g$)—primarily obtained from well test and MPFM measurements. 
The independent variables comprise pressure, temperature, and operational parameters that are physically linked to well inflow and surface flow behavior.

\subsection{Independent and Reference Variables}
\label{subsec:dataset_variables}

The raw dataset for each well contains a combination of wellhead measurements, downhole
measurements, choke parameters, well test rate estimations and MPFM rate estimations. 
Table~\ref{tab:independent_vars} summarizes the independent operational variables of the dataset, while Table~\ref{tab:geometrical_const} lists the geometrical parameters associated with each well.
Although geometrical parameters such as tubing diameter, measured depth, and completion geometry play a fundamental role in governing multiphase flow behavior, they are \textbf{not used as independent input variables} in the proposed modeling architecture. This is because the operational measurements—such as pressures, temperatures, and choke settings—are themselves physical responses of the well system that inherently embed the effects of the underlying geometry. Consequently, explicitly including geometrical parameters as model inputs would introduce redundancy and may lead to identifiability issues or spurious correlations, particularly under sparse and irregular well test supervision.
Instead, geometrical data are used exclusively to provide \textbf{initial estimates and constraints during the physics-based model calibration stage}. These parameters inform the initialization of flow equations and serve as fixed structural inputs to the physics model, while the operational variables act as the primary explanatory signals for dynamic rate prediction. This separation ensures physical consistency, reduces model complexity, and improves robustness of the hybrid framework when applied across wells with limited historical data.
 

\begin{table}[!htb]
\uomTable{Indepdent operational variables}{tab:independent_vars}{\begin{tabularx}{0.9\textwidth}{lYYYY} 
\hline
Variable & Description & Unit \\
\hline
dhp & Downhole pressure & bara \\
dht & Downhole temperature & \textdegree C \\
whp & Wellhead pressure & bara \\
wht & Wellhead temperature & \textdegree C \\
dcp & Downstream choke pressure & bara \\
choke & Downstream choke regulation & \% \\
well\_id & Well unique id & unitless (string) \\
\hline
\end{tabularx}}{}
\end{table}

\begin{table}[!htb]
\uomTable{Geometrical static data}{tab:geometrical_const}{\begin{tabularx}{0.9\textwidth}{lYYYY} 
\hline
Variable & Description & Unit \\
\hline
dh\_md & Downhole measured depth & m \\
dh\_tvd & Downhole true vertical depth & m \\
wh\_md & Wellhead measured depth & m \\
wh\_tvd & Wellhead true vertical depth & m \\
\hline
\end{tabularx}}{}
\end{table}

Table~\ref{tab:reference_vars} summarizes the reference ground truth variables of the dataset. 
A key modelling decision is the use of \textbf{well test measurements} ($q_o^{\mathrm{WT}}$,
$q_g^{\mathrm{WT}}$, $q_w^{\mathrm{WT}}$) as the primary ground-truth supervisory targets,
rather than MPFM flow rates. This choice is justified as follows:

\begin{itemize}
    \item Well tests provide \textbf{officially calibrated} and \textbf{field-accepted} flow-rate
          measurements, typically validated using test separator systems.
    \item MPFM measurements, while continuous, can exhibit \textbf{bias, drift, salinity sensitivity},
          and \textbf{phase inversion errors}, especially under varying water-cut and pressure conditions.
    \item For long-term production modelling and scientific evaluation,
          well-test values serve as a more reliable \textbf{ground-truth reference}.
\end{itemize}

As a consequence, MPFM measurements are excluded from model training while all available operational variables are retained as independent features.


\begin{table}[!htb]
\uomTable{Reference ground truth variables}{tab:reference_vars}{\begin{tabularx}{0.9\textwidth}{lYYYY} 
\hline
Variable & Description & Unit & Source \\
\hline
qo\_well\_test & Well test measured oil rate & Sm$^3$/h & Well test \\
qg\_well\_test & Well test measured gas rate & Sm$^3$/h & Well test \\
qw\_well\_test & Well test measured water rate & Sm$^3$/h & Well test \\
\hline
\end{tabularx}}{}
\end{table}

\begin{table}[!htb]
\uomTable{MPFM estimated rates}{tab:mpfm_vars}{\begin{tabularx}{0.9\textwidth}{lYYYY} 
\hline
Variable & Description & Unit & Source \\
\hline
qo\_mpfm & MPFM measured oil rate & Sm$^3$/h & MPFM \\
qg\_mpfm & MPFM measured gas rate & Sm$^3$/h & MPFM \\
qw\_mpfm & MPFM measured water rate & Sm$^3$/h & MPFM \\
\hline
\end{tabularx}}{}
\end{table}

Table~\ref{tab:dataset_row_counts} summarizes the availability and temporal coverage of historical well test measurements used in this study. 
A total of 834 well test records are available across seven production wells, spanning a period from March 2017 to August 2023. 
The number of records per well varies substantially, reflecting differences in operational practices, testing frequency, and data availability over the field lifecycle. 
Such sparsity and irregularity are characteristic of well test data, as production tests are typically conducted intermittently due to operational constraints, cost considerations, and production priorities, rather than being continuously acquired like sensor-based measurements. 
Consequently, the resulting dataset exhibits limited sample sizes and non-uniform temporal spacing, which poses challenges for data-driven modeling approaches. 
This motivates the adoption of physics-based and hybrid modeling strategies in this work, which are better suited to learning from sparse, irregular, and heterogeneous well test observations.

\begin{table}[h!]
\uomTable{Summary of available historical measurements for each production well}{tab:dataset_row_counts}{\begin{tabularx}{0.9\textwidth}{lYYYY} 
\hline
Well ID & Number of Records & Start Date & End Date \\
\hline
W06 & 137 & 2017-03-06 & 2023-05-07 \\
W08 & 69 & 2018-07-30 & 2023-06-03 \\
W10 & 222 & 2017-03-13 & 2023-07-31 \\
W11 & 154 & 2017-03-01 & 2023-08-04 \\
W15 & 86 & 2017-03-11 & 2023-05-13 \\
W18 & 89 & 2018-05-03 & 2023-08-09 \\
W19 & 77 & 2017-09-02 & 2023-07-27 \\
\hline
\textbf{All} & \textbf{834} & 2017-03-01 & 2023-08-09 \\
\hline
\end{tabularx}}{}
\end{table}

Downhole pressure and temperature measurements, shown in Figures~\ref{fig:W10_dhp} and~\ref{fig:W10_dht}, respectively, represent the thermodynamic conditions at the producing interval. These variables are critical for characterizing reservoir drawdown, inflow behavior, and fluid properties, and they directly influence the physics-based inflow and vertical lift formulations employed in the model.
Figures~\ref{fig:W10_whp} and~\ref{fig:W10_wht} illustrate the wellhead pressure and temperature measurements. These measurements capture surface operating conditions and reflect the combined effects of reservoir inflow, multiphase flow in the wellbore, and surface choke control. Wellhead measurements are typically more frequently available and are therefore particularly valuable for continuous production monitoring and data-driven model calibration.
The downstream choke pressure and choke regulation signals are shown in Figures~\ref{fig:W10_dcp} and~\ref{fig:W10_choke}. These variables describe the imposed flow restriction and surface control strategy applied to the well. Variations in choke opening and downstream pressure directly affect the pressure drawdown across the choke and, consequently, the produced flow rates. As such, these measurements play an important role in capturing operational transients and control-induced production variability.

Collectively, the measurements presented in Figures~\ref{fig:W10_dhp}--\ref{fig:W10_choke} constitute the primary operational dataset used for model training, calibration, and validation. Their combined use enables the hybrid modeling framework to exploit both physics-based consistency and data-driven adaptability under varying operating conditions.

%------------------------------------------------
% W10 – Downhole measurements
%------------------------------------------------
\begin{figure}[!htb]
\centering
\begin{minipage}{0.48\textwidth}
    \uomFig{W10 -- Downhole pressure}{fig:W10_dhp}
    {\includegraphics[width=\linewidth]{images/W10_dhp.png}}{}
\end{minipage}\hfill
\begin{minipage}{0.48\textwidth}
    \uomFig{W10 -- Downhole temperature}{fig:W10_dht}
    {\includegraphics[width=\linewidth]{images/W10_dht.png}}{}
\end{minipage}
\end{figure}

%------------------------------------------------
% W10 – Wellhead measurements
%------------------------------------------------
\begin{figure}[!htb]
\centering
\begin{minipage}{0.48\textwidth}
    \uomFig{W10 -- Wellhead pressure}{fig:W10_whp}
    {\includegraphics[width=\linewidth]{images/W10_whp.png}}{}
\end{minipage}\hfill
\begin{minipage}{0.48\textwidth}
    \uomFig{W10 -- Wellhead temperature}{fig:W10_wht}
    {\includegraphics[width=\linewidth]{images/W10_wht.png}}{}
\end{minipage}
\end{figure}

%------------------------------------------------
% W10 – Downstream choke measurements
%------------------------------------------------
\begin{figure}[!htb]
\centering
\begin{minipage}{0.48\textwidth}
    \uomFig{W10 -- Downstream choke pressure}{fig:W10_dcp}
    {\includegraphics[width=\linewidth]{images/W10_dcp.png}}{}
\end{minipage}\hfill
\begin{minipage}{0.48\textwidth}
    \uomFig{W10 -- Downstream choke regulation}{fig:W10_choke}
    {\includegraphics[width=\linewidth]{images/W10_choke.png}}{}
\end{minipage}
\end{figure}


\subsection{Well-Level Preprocessing}

\begin{itemize}
    \item \textbf{Well-wise data isolation and temporal ordering:}  
    The preprocessing is performed independently for each well by isolating records associated with a single well identifier. Observations are sorted in chronological order to preserve the original temporal structure of the data.

    \item \textbf{Removal of physically invalid sensor readings:}  
    To ensure physical consistency and numerical stability, observations containing zero-valued measurements in key operational variables are removed. These include downhole pressure, wellhead pressure, downstream choke pressure, downhole temperature, wellhead temperature, and choke opening. Zero values are interpreted as sensor dropouts, invalid measurements, or non-operational conditions.

    \item \textbf{Consistent well identification and encoding:}  
    The well identifier is explicitly reassigned to each processed subset to ensure consistency. In addition, a numerical well code is generated from the categorical well identifier to enable machine-learning models to condition on well identity in a numerically stable manner.

    \item \textbf{Normalization of choke opening:}  
    The choke opening is converted from a percentage representation to a normalized fraction by dividing by 100. This transformation ensures numerical consistency and improves convergence behavior in downstream machine-learning components.

    \item \textbf{Removal of incomplete observations:}  
    Observations containing missing values in either dependent variables or independent input variables are removed. This step ensures that all retained samples are fully defined for model training, calibration, and validation.

    \item \textbf{Variable selection and schema enforcement:}  
    After preprocessing, the dataset is restricted to a predefined set of required variables. This enforces a consistent data schema across all wells and prevents unintended variables from entering the modeling pipeline.

    \item \textbf{Enforcement of physically admissible flow rates:}  
    Observations with negative well-test oil, gas, or water rates are excluded to maintain physical realism and prevent numerical instability during model fitting.

    \item \textbf{Treatment of zero well-test water rates:}  
    Reported zero water rates in well-test data may correspond either to genuinely water-free production or to measurements below the detection resolution of the test separator. To mitigate numerical instability and distortion of error metrics, zero water-rate values occurring after the onset of water production for a given well are treated as left-censored observations and replaced with the minimum non-zero water rate reported for that well. Zero water-rate values prior to the first observed water production are retained unchanged.

    \item \textbf{Exclusion of non-producing operating states:}  
    Rows corresponding to non-producing conditions, identified by simultaneous zero oil and gas well-test rates, are removed. These conditions are not representative of active production behavior and are therefore excluded from the modeling dataset.
\end{itemize}


\subsection{Derivation of Multiphase Flow Variables}

Additional flow-related variables are derived from the available MPFM measurements solely for the purpose of evaluating MPFM estimates against well-test data, and are not used for physics-model calibration or hybrid model training.
The gas--oil ratio (GOR) is computed as:
\begin{equation}
\text{gor\_mpfm} = \frac{qg\_mpfm}{qo\_mpfm},
\end{equation}
where $gor\_mpfm$ and $qg\_mpfm$ denote the MPFM measured gas and oil flow rates, respectively.

The water flow rate is reconstructed from the measured water cut using the relationship
\begin{equation}
qw\_mpfm = \frac{\text{wc\_mpfm} \cdot qo\_mpfm}{1 - \text{wc\_mpfm}},
\end{equation}
where $\text{wc}$ is the water cut expressed as a fraction. This formulation ensures consistency between oil, water, and water cut measurements and avoids reliance on potentially noisy direct water rate measurements.

\subsection{Temporal Index Construction}

Following well-level preprocessing, a unified temporal index is constructed to enable consistent time-series modeling across wells. A reference time is defined as the earliest timestamp present in the dataset. For each observation, a discrete time index is computed as
\begin{equation}
\text{time\_idx} =
\left\lfloor
\frac{t - t_{\text{ref}}}{\Delta t}
\right\rfloor,
\end{equation}
where $t$ is the observation timestamp, $t_{\text{ref}}$ is the reference time, and $\Delta t$ corresponds to the base time resolution of the data. This integer-valued index provides a compact and model-friendly representation of time while preserving relative temporal ordering.

\subsection{Cross-Well Consolidation and Encoding}

After preprocessing individual wells, the processed datasets are concatenated into a single multi-well dataframe. To support machine-learning models that require numerical inputs, categorical well identifiers are encoded into continuous numerical codes using category-based indexing. This encoding preserves well identity information while remaining compatible with standard regression models.

The final output of the preprocessing stage is a cleaned, physically consistent, and temporally ordered dataset containing all required input variables, derived flow quantities, and auxiliary identifiers required for subsequent physics-based modeling and machine-learning residual learning.


\section{Physics-Based Inflow Modelling}
\label{sec:physics_model}

The physics-based inflow model provides the first-principles foundation for predicting
oil, water, and gas flow rates using measurable wellhead and downhole variables.
This model serves as the structured prior within the hybrid learning framework.
It maps the instantaneous well state - including downhole pressure, flowing temperature,
choke opening, and pressure drop - to a set of physically constrained rate predictions.
The model consists of:

\begin{enumerate}
    \item A pressure-normalized liquid-rate inflow equation that models the total produced liquid rate as a nonlinear function of flowing downhole pressure relative to an estimated reservoir pressure.
    \item A logistic water-cut closure that maps choke setting, pressure, and temperature variables to a bounded water fraction, ensuring physically consistent phase partitioning of the total liquid rate.
    \item A pressure-driven gas-rate formulation in which gas production is governed by the square-root pressure drawdown and a nonlinear choke effectiveness function, capturing flow control and gas deliverability effects.
    \item A parameter calibration procedure based on constrained nonlinear least-squares calibration, jointly fitting liquid, water, and gas rate observations through normalized residual minimization.
\end{enumerate}

\subsection{Modeling Philosophy}
In practical production environments, detailed mechanistic multiphase flow models require extensive fluid property data, well geometry, and calibration effort, which may not be consistently available for all wells. Moreover, such models may still suffer from uncertainty under changing operating conditions. Therefore, a simplified physics-based formulation is adopted, prioritizing robustness, interpretability, and compatibility with sparse supervision.
The physics-based model is designed to:

\begin{itemize}
    \item Capture first-order relationships between pressure, choke setting, and flow rates
    \item Maintain physical plausibility across operating regimes
    \item Provide smooth and stable predictions suitable for residual learning
    \item Allow independent calibration for each well
\end{itemize}

\subsection{Model Assumptions and Physical Basis}
\label{subsec:model_assumptions}

The model assumes that the flowing downhole pressure,
\(
P_{\mathrm{wf}}(t)
\),
dominantly governs the total liquid production rate from the well.
The reservoir is represented using an effective reservoir pressure
\(
P_{\mathrm{res}}
\),
treated as an unknown parameter to be estimated from the data.
The following assumptions are embedded in the formulation:

\begin{itemize}
    \item The total liquid rate \(q_L(t)\) follows a quadratic inflow relationship
        with respect to the normalized flowing downhole pressure \(P_{\mathrm{wf}}/P_{\mathrm{res}}\).
    \item Water cut is determined by operational variables
        (choke opening, pressure drop, temperatures),
        using a logistic regression mapping to ensure \(0 \le \mathrm{wc}(t) \le 1\).
    \item Gas production is driven by a pressure drawdown term proportional to 
        \( \sqrt{P_{res} - P_{wf}} \), modulated by a nonlinear choke effectiveness function, 
        representing flow through a restriction under choked or near-choked conditions.
    \item Oil, water, and gas rates remain non-negative due to clipping operations.
\end{itemize}

All relationships act instantaneously and do not assume temporal memory,
which enables the model to make predictions at any arbitrary timestamp
given the set of independent well measurements.

\subsection{Liquid-Rate Inflow Equation}
\label{subsec:liquid_rate}

Total liquid production is modeled using a pressure-driven inflow relationship inspired by classical inflow performance relationships. For wells operating below bubble-point pressure or under multiphase flow conditions, quadratic inflow formulations provide a reasonable approximation of nonlinear pressure–rate behavior while remaining computationally simple.
The total liquid rate is expressed as:

\begin{equation}
q_L = q_{L,\max}\left(1 - a \frac{P_{wf}}{P_{res}} - b \left(\frac{P_{wf}}{P_{res}}\right)^2 \right),
\end{equation}

where:
\begin{itemize}
    \item $q_L$ is the total liquid (oil + water) production rate
    \item $q_{L,\max}$ represents the theoretical maximum liquid rate
    \item $P_{wf}$ is the downhole flowing pressure
    \item $P_{res}$ is an effective reservoir pressure
    \item $a$ and $b$ are empirical coefficients calibrated per well
\end{itemize}

This formulation captures the nonlinear decline in liquid production with increasing flowing pressure while remaining flexible enough to represent a wide range of well behaviors. The quadratic term allows the model to accommodate deviations from ideal Darcy flow caused by multiphase effects and near-wellbore pressure losses.


\subsection{Logistic Water-Cut Closure Model}
\label{subsec:wc_closure}

Phase splitting between oil and water is achieved using a logistic water-cut closure. Logistic functions are commonly employed in petroleum engineering to model bounded variables such as water cut due to their smoothness and asymptotic behavior.
Water cut is defined as:

\begin{equation}
wc = \sigma\left(\mathbf{X}_{wc} \mathbf{A}\right),
\end{equation}

where:
\begin{itemize}
    \item $wc$ is the water cut ($0 \le wc \le 1$)
    \item $\sigma(\cdot)$ denotes the logistic function
    \item $\mathbf{X}_{wc}$ is a feature vector constructed from pressures, temperatures, and choke-related variables
    \item $\mathbf{A}$ is a vector of learnable coefficients
\end{itemize}

The logistic formulation ensures that predicted water cut remains physically bounded and provides a smooth transition between oil-dominated and water-dominated flow regimes. The inclusion of operational variables allows the model to reflect changes in phase behavior due to pressure drawdown, thermal effects, and choke regulation.
Oil and water production rates are then computed as:

\begin{equation}
q_w = wc \cdot q_L, \qquad q_o = (1 - wc) \cdot q_L.
\end{equation}

This formulation enforces mass balance consistency between oil, water, and total liquid production.

\subsection{Gas Flow Modelling}
\label{subsec:gas_rate}

Gas production is modeled using a pressure-drop-driven relationship modulated by choke effectiveness. In producing wells, gas flow is strongly influenced by reservoir pressure drawdown and surface flow restrictions imposed by choke settings.
The gas rate is modeled as:

\begin{equation}
q_g = C_g \sqrt{P_{res} - P_{wf}} \cdot \sigma\!\left(k_{ch}(ch - ch_0)\right),
\end{equation}

where:
\begin{itemize}
    \item $q_g$ is the gas production rate
    \item $C_g$ is a gas flow coefficient
    \item $ch$ denotes the choke opening
    \item $k_{ch}$ and $ch_0$ control choke sensitivity and activation threshold
\end{itemize}

The square-root pressure dependency is motivated by compressible gas flow through restrictions, while the logistic choke term captures the nonlinear influence of choke opening on gas throughput. This formulation provides stable behavior across a wide range of choke settings and avoids discontinuities associated with piecewise choke models.

\subsection{Parameter Calibration via Nonlinear Least Squares}
\label{subsec:parameter_calibration}

All physics model parameters are calibrated independently for each well using nonlinear least-squares optimization against available multiphase flow meter measurements. Calibration is performed jointly across oil, water, and gas rates using normalized residuals to ensure balanced fitting across phases.
Physically plausible bounds are imposed on all parameters to prevent non-physical solutions and improve numerical stability. This per-well calibration strategy allows the model to adapt to well-specific characteristics such as productivity, completion design, and reservoir conditions while maintaining a consistent functional form across wells.
The objective of the parameter estimation procedure is to identify physically plausible model parameters that minimize the discrepancy between physics-based flow predictions and available ground-truth flow measurements obtained from multiphase flow meters.

\subsubsection{Problem Formulation}

Let $\boldsymbol{\theta}$ denote the vector of physics model parameters to be estimated for a given well. This parameter vector includes inflow coefficients, water-cut closure parameters, and gas flow coefficients. For a set of $N$ observation times at which measured flow rates are available, the physics model produces predicted oil, water, and gas rates:
\begin{equation}
\hat{\mathbf{q}}_i(\boldsymbol{\theta}) =
\begin{bmatrix}
\hat{q}_{o,i}(\boldsymbol{\theta}) \\
\hat{q}_{w,i}(\boldsymbol{\theta}) \\
\hat{q}_{g,i}(\boldsymbol{\theta})
\end{bmatrix},
\qquad i = 1, \ldots, N,
\end{equation}
while the corresponding measured flow rates are denoted by:
\begin{equation}
\mathbf{q}_i =
\begin{bmatrix}
q_{o,i} \\
q_{w,i} \\
q_{g,i}
\end{bmatrix}.
\end{equation}

The residual vector at each observation time is defined as the difference between measured and predicted flow rates:
\begin{equation}
\mathbf{r}_i(\boldsymbol{\theta}) = \mathbf{q}_i - \hat{\mathbf{q}}_i(\boldsymbol{\theta}).
\end{equation}

\subsubsection{Weighted Least Squares Objective Function}

To account for differences in magnitude and measurement uncertainty across phases, a weighted least squares objective function is employed. The total cost function is formulated as:
\begin{equation}
J(\boldsymbol{\theta}) = \sum_{i=1}^{N} \mathbf{r}_i(\boldsymbol{\theta})^\top \mathbf{W} \, \mathbf{r}_i(\boldsymbol{\theta}),
\end{equation}
where $\mathbf{W}$ is a diagonal weighting matrix:
\begin{equation}
\mathbf{W} =
\begin{bmatrix}
w_o & 0 & 0 \\
0 & w_w & 0 \\
0 & 0 & w_g
\end{bmatrix}.
\end{equation}

The weights $w_o$, $w_w$, and $w_g$ are selected to balance the contribution of oil, water, and gas residuals, preventing any single phase from dominating the optimization due to scale differences or higher variability.

\subsubsection{Nonlinear Least Squares Optimization}

The parameter estimation problem is posed as a constrained nonlinear least squares optimization:
\begin{equation}
\boldsymbol{\theta}^* = \arg \min_{\boldsymbol{\theta}} J(\boldsymbol{\theta}),
\end{equation}
subject to physically motivated bounds:
\begin{equation}
\boldsymbol{\theta}_{\min} \le \boldsymbol{\theta} \le \boldsymbol{\theta}_{\max}.
\end{equation}

All parameters are estimated via nonlinear least-squares calibration against well-test data and represent effective coefficients that compensate for unmodeled multiphase flow, surface interactions, and operational variability.
These bounds enforce physical plausibility, such as non-negative productivity indices, reasonable reservoir pressures, and monotonic choke response, and significantly improve numerical stability and convergence behavior.

The optimization is solved using a trust-region-based nonlinear least squares algorithm, which iteratively linearizes the residual function around the current parameter estimate and updates the parameters by solving a sequence of local quadratic subproblems. This approach is well-suited for moderately nonlinear models and provides robust convergence under noisy and sparse data conditions.
The set of model parameters calibrated during physics-model fitting, together with their physical interpretation and units, is summarized in Table~\ref{tab:physics_calibrated_parameters}.

\begin{table}[!htb]
\uomTable{Parameters and initial guesses used for nonlinear calibration of the physics-based well flow model}{tab:physics_calibrated_parameters}{\begin{tabularx}{1.0\textwidth}{lYYYY} 
\hline
\textbf{Parameter} & \textbf{Initial guess} & \textbf{Unit} & \textbf{Physical interpretation} \\
\hline
$P_{\mathrm{res}}$ 
& $P_{\mathrm{wf,max}} + \Delta P$ 
& bara
& Effective reservoir pressure inferred from flowing pressure with hydrostatic or heuristic offset \\
$q_{L,\max}$ 
& $\overline{(q_o + q_w)}$ 
& Sm$^3$/h 
& Maximum liquid production capacity (lumped inflow strength) \\
$a$ 
& $0.2$ 
& -- 
& Linear pressure sensitivity coefficient in the liquid inflow relationship \\
$b$ 
& $0.5$ 
& -- 
& Quadratic pressure sensitivity coefficient accounting for nonlinearity \\
$C_g$ 
& $50.0$ 
& Sm$^3$/h/$\sqrt{\text{bar}}$ 
& Reservoir gas productivity coefficient in the pressure-driven gas-rate formulation \\
$k_{\mathrm{ch}}$ 
& $5.0$ 
& -- 
& Choke sensitivity parameter controlling the sharpness of the logistic choke response \\
$\mathrm{ch}_0$ 
& $0.3$ 
& -- 
& Reference (midpoint) choke opening for transition from restricted to open flow \\
$C_{\mathrm{gl}}$ 
& $0.01$ 
& -- 
& Gas-lift efficiency coefficient scaling injected gas contribution \\
$A_{\mathrm{wc},1\ldots8}$ 
& $0.0$ 
& -- 
& Initial coefficients of the logistic water-cut closure model \\
\hline
\end{tabularx}}{}
\end{table}


\subsubsection{Initialization and Regularization}

Initial parameter estimates are chosen based on engineering heuristics and prior operational knowledge, such as typical reservoir pressure ranges and expected choke sensitivities. Careful initialization reduces the risk of convergence to non-physical local minima.
To mitigate overfitting under sparse measurement conditions, implicit regularization is achieved through:

\begin{itemize}
    \item Bounded parameter constraints
    \item Joint calibration across all available phases
    \item Smooth functional forms in the physics model
\end{itemize}

Explicit regularization terms are not introduced to preserve the interpretability of calibrated parameters.

\subsubsection{Handling Sparse and Irregular Measurements}

The nonlinear least squares formulation naturally accommodates sparse and irregular measurement schedules, as the objective function is evaluated only at timestamps where flow rate measurements are available. No interpolation or artificial densification of measured flow data is performed during calibration, thereby avoiding bias and preserving the integrity of the observed data.

\subsubsection{Calibration Quality Assessment}

Following optimization, calibration quality is assessed using residual diagnostics and goodness-of-fit metrics computed on the measured data. Visual inspection of predicted versus measured flow rates and residual time series is also performed to identify systematic biases or regime-dependent errors.

Importantly, the calibrated physics-based model is not expected to achieve perfect agreement with measured data. Instead, it is intended to capture dominant flow behavior while leaving systematic discrepancies to be addressed by the subsequent residual learning component.

\subsubsection{Role in the Hybrid Modeling Architecture}

The independently calibrated physics-based model serves as a physically grounded baseline within the hybrid architecture. By constraining the solution space through first-principles relationships and calibrated parameters, the nonlinear least squares estimation stage enables the residual learning model to focus on compensating for unmodeled dynamics, thereby improving overall prediction accuracy, robustness, and data efficiency.

\section{Machine-Learning Residual Modelling}
\label{sec:residual_ml}

The physics-based inflow model provides physically consistent baseline predictions for oil,
water, and gas production. However, real production behaviour is often influenced by
complex wellbore dynamics, operational disturbances, multiphase interactions, and other
effects that are not explicitly captured by the simplified analytical physics model.
To account for these discrepancies, a machine learning correction stage is introduced.
This stage learns the residual structure between observed rates and their corresponding
physics-based predictions, producing a hybrid physics - ML model with substantially improved
accuracy and generalisation.

The machine-learning residual subsystem operates on a feature set derived from instantaneous measurements and lagged historical values, and predicts additive corrections in logarithmic space for oil, gas, and water flow rates.


\subsection{Residual Learning Model Philosophy}
\label{subsec:residual_learning_model_philosophy}

The use of residual learning instead of direct rate prediction provides several advantages:

\begin{itemize}
    \item \textbf{Preservation of physical structure:}  
          The physics model enforces consistency between pressure, choke settings,
          phase split, liquid rate, and gas rate. By learning residuals, the machine-learning
          model augments rather than replaces physically plausible behaviour.

    \item \textbf{Reduction of model complexity:}  
          The machine-learning component learns only deviations from the physics-based
          predictions, which are typically smoother and of smaller magnitude. This reduces
          the effective variance of the learned model.

    \item \textbf{Improved generalisation:}  
          Because the residual model does not need to learn fundamental inflow and choke
          relationships, it exhibits improved generalisation to unseen operating conditions
          and wells.

    \item \textbf{Enhanced interpretability:}  
          The learned residuals provide direct insight into where and when the physics-based
          model diverges from observed data, thereby supporting diagnostic analysis and model
          refinement.

    \item \textbf{Enhanced scalability:}  
        The residual learning formulation supports scalable deployment across large well
        populations by decoupling well-specific physics calibration from global data-driven
        learning. Physics models are calibrated independently for each well, while a single
        machine-learning residual model is shared across all wells. This design significantly
        reduces training and inference overhead, enabling efficient application of virtual
        flow metering in assets comprising thousands of wells.

    \item \textbf{Improved robustness under sparse and irregular measurements:}  
        The residual learning formulation leverages physics-based predictions as a stable
        baseline, allowing reliable flow-rate estimation even when ground-truth measurements
        are sparse, irregular, or intermittently unavailable. By learning residuals only
        where reference measurements exist and pooling information across multiple wells,
        the framework reduces sensitivity to data gaps and supports robust operation under
        realistic field data constraints.
\end{itemize}

Overall, the machine-learning residual modelling component serves as a complementary layer that enhances physics-based predictions while respecting physical structure and industrial deployment requirements.

\subsection{Residual Learning Formulation}

Let $\hat{q}_o^{\text{phys}}(t)$, $\hat{q}_w^{\text{phys}}(t)$, and $\hat{q}_g^{\text{phys}}(t)$ denote the oil, water, and gas flow rates predicted by the calibrated physics model at time $t$. The corresponding measured flow rates obtained from multiphase flow meters are denoted by $q_o(t)$, $q_w(t)$, and $q_g(t)$.

Residual learning is performed in logarithmic space to stabilize variance and ensure positivity of predicted flow rates. The residual targets are defined as
\begin{equation}
r_i(t) = \log\!\left(1 + q_i(t)\right) - \log\!\left(1 + \hat{q}_i^{\text{phys}}(t)\right),
\quad i \in \{o, w, g\}.
\end{equation}

This formulation allows the residual model to learn multiplicative corrections in the original flow-rate domain while remaining numerically stable for low-rate conditions.

\subsection{Feature Construction}

The input feature vector for residual learning is constructed by concatenating measured operating variables with physics-based flow predictions. For each valid time index $t$, the feature vector is defined as
\begin{equation}
\mathbf{x}(t) =
\begin{bmatrix}
\mathbf{u}(t) \\
\hat{q}_o^{\text{phys}}(t) \\
\hat{q}_w^{\text{phys}}(t) \\
\hat{q}_g^{\text{phys}}(t)
\end{bmatrix},
\end{equation}
where $\mathbf{u}(t)$ represents the set of independent well measurements, including pressures, temperatures, choke settings, and lagged variables where applicable.

Including physics-based predictions as explicit features enables the residual model to condition its corrections on physically meaningful state estimates, thereby reducing extrapolation risk and improving generalization.

\subsection{Data Normalization and Polynomial Expansion}

Prior to training, the feature matrix is standardized using zero-mean and unit-variance scaling,
\begin{equation}
\tilde{\mathbf{x}} = \frac{\mathbf{x} - \boldsymbol{\mu}}{\boldsymbol{\sigma}},
\end{equation}
where $\boldsymbol{\mu}$ and $\boldsymbol{\sigma}$ are computed from the training dataset.

To capture low-order nonlinear interactions between features, a polynomial feature expansion of degree $d$ is applied,
\begin{equation}
\mathbf{z} = \Phi_d(\tilde{\mathbf{x}}),
\end{equation}
where $\Phi_d(\cdot)$ denotes the polynomial feature mapping without bias terms. In the current implementation, a linear expansion ($d=1$) is used to maintain computational efficiency and avoid overfitting under sparse data conditions.

\subsection{Residual Model Architecture}

The residual mapping from feature space to residual corrections is learned using a multi-output regression framework,
\begin{equation}
\mathbf{r}(t) = f_{\text{ML}}(\mathbf{z}(t)),
\end{equation}
where $\mathbf{r}(t) = [r_o(t), r_w(t), r_g(t)]^\top$.

The function $f_{\text{ML}}(\cdot)$ is implemented as a \textit{MultiOutputRegressor} with gradient-boosted decision trees as base learners. Specifically, a histogram-based gradient boosting regressor is employed due to its robustness, fast training, and suitability for tabular industrial data. Key hyperparameters are selected to balance accuracy and computational efficiency, including shallow tree depth, early stopping, and minimum leaf size constraints.

A single global residual model is trained using data pooled from all wells, while physics models are calibrated independently for each well. This design enables cross-well learning of common residual patterns while preserving well-specific physical behavior.

\subsection{Hybrid Prediction and Residual Application}

During inference, physics-based predictions are first generated for each well. The trained residual model then produces residual estimates $\hat{r}_i(t)$, which are applied selectively depending on the flow phase and operating regime.

For oil and gas, residual corrections are applied in logarithmic space as
\begin{equation}
\hat{q}_i(t) =
\exp\!\left(
\log\!\left(1 + \hat{q}_i^{\text{phys}}(t)\right) + \hat{r}_i(t)
\right) - 1,
\quad i \in \{o, g\}.
\end{equation}

For water, residual corrections are applied only when the physics-based water rate exceeds a predefined threshold $q_w^{\text{min}}$, ensuring that spurious corrections are not introduced during dry or near-dry conditions,
\begin{equation}
\hat{q}_w(t) =
\begin{cases}
\exp\!\left(
\log\!\left(1 + \hat{q}_w^{\text{phys}}(t)\right) + \hat{r}_w(t)
\right) - 1, & \hat{q}_w^{\text{phys}}(t) > q_w^{\text{min}}, \\
0, & \text{otherwise}.
\end{cases}
\end{equation}

Additional gating rules based on water cut and residual magnitude are applied to oil predictions to prevent physically implausible corrections under extreme conditions.

\subsection{Consistency Constraints}

To preserve physical consistency, predicted oil and water rates are constrained such that
\begin{equation}
\hat{q}_o(t) + \hat{q}_w(t) = \hat{q}_L(t),
\end{equation}
where $\hat{q}_L(t)$ denotes the total liquid rate. Water predictions are clipped to ensure non-negativity and boundedness, and oil rates are adjusted accordingly.


\section{Hybrid Physics--ML Rate Synthesis}
\label{sec:hybrid-rate-synthesis}

The hybrid physics--machine learning (ML) rate synthesis component defines how
physics-based predictions and machine-learning residual corrections are combined
to produce the final estimates of oil, water, and gas flow rates. This stage
constitutes the final inference step of the proposed architecture and is critical
for ensuring physical consistency, robustness, and safe deployment under sparse
and irregular measurement conditions.

\subsection{Sequential Inference Structure}

The proposed framework follows a strictly sequential inference strategy. For each
well and time index $t$, physics-based predictions are generated first, followed by
selective application of machine-learning residual corrections. Let
\begin{equation}
\hat{q}_o^{P}(t), \; \hat{q}_w^{P}(t), \; \hat{q}_g^{P}(t)
\end{equation}
denote the oil, water, and gas flow rates predicted by the calibrated physics model,
and let
\begin{equation}
\hat{r}_o(t), \; \hat{r}_w(t), \; \hat{r}_g(t)
\end{equation}
denote the corresponding residuals predicted by the machine-learning model.

The physics model is always evaluated regardless of data availability, while the
machine-learning component operates as a corrective layer whose influence is
controlled through physically motivated gating mechanisms.

\subsection{Log-Space Residual Application}

Residuals are learned and applied in logarithmic space to ensure numerical stability
and preserve non-negativity of flow rates. For a given phase $i \in \{o, w, g\}$,
the corrected rate is computed as
\begin{equation}
\hat{q}_i(t)
=
\exp\!\left(
\log\!\left(1 + \hat{q}_i^{P}(t)\right) + \hat{r}_i(t)
\right) - 1.
\end{equation}

This formulation corresponds to a multiplicative correction in the original rate
domain and reflects the assumption that modeling errors scale with the magnitude
of the predicted flow rate.

\subsection{Phase-Specific Synthesis Rules}

The hybrid synthesis logic differs across fluid phases to reflect their distinct
flow characteristics and data reliability.

\subsubsection{Oil Rate Synthesis}

Oil rate corrections are applied conditionally. The physics-based oil prediction
$\hat{q}_o^{P}(t)$ is retained unless one or more gating conditions are satisfied.
These conditions include elevated water cut or large predicted residual magnitude.
When gating is active, the corrected oil rate is computed using log-space residual
application:
\begin{equation}
\hat{q}_o(t)
=
\exp\!\left(
\log\!\left(1 + \hat{q}_o^{P}(t)\right) + \hat{r}_o(t)
\right) - 1.
\end{equation}
If gating conditions are not satisfied, the physics-based oil rate is preserved.

\subsubsection{Gas Rate Synthesis}

Gas rates are corrected unconditionally using the residual model, reflecting the
strong dependence of gas flow on pressure drawdown and choke behavior:
\begin{equation}
\hat{q}_g(t)
=
\exp\!\left(
\log\!\left(1 + \hat{q}_g^{P}(t)\right) + \hat{r}_g(t)
\right) - 1.
\end{equation}
The corrected gas rate is subsequently clipped to ensure non-negativity.

\subsubsection{Water Rate Synthesis}

Water rate corrections are applied only when the physics-based water rate exceeds
a predefined minimum threshold $q_w^{\min}$:
\begin{equation}
\hat{q}_w(t)
=
\begin{cases}
\exp\!\left(
\log\!\left(1 + \hat{q}_w^{P}(t)\right) + \hat{r}_w(t)
\right) - 1,
& \hat{q}_w^{P}(t) > q_w^{\min}, \\[6pt]
0,
& \text{otherwise}.
\end{cases}
\end{equation}

This gating strategy prevents spurious corrections during dry or near-dry conditions
where measurement noise and model uncertainty are high.

\subsection{Physical Consistency Enforcement}

Following residual application, additional consistency constraints are enforced to
ensure physically meaningful predictions. The total liquid rate is defined as
\begin{equation}
\hat{q}_L(t) = \hat{q}_o(t) + \hat{q}_w(t).
\end{equation}

Water rates are clipped to ensure
\begin{equation}
0 \le \hat{q}_w(t) \le \hat{q}_L(t),
\end{equation}
and the oil rate is adjusted accordingly to preserve liquid consistency. All phase
rates are constrained to be non-negative.

\subsection{Role of Gating and Safety Mechanisms}

The synthesis stage incorporates multiple gating and clipping mechanisms to limit
machine-learning influence in poorly observed regimes. These safeguards reduce
extrapolation risk under sparse or irregular measurements and ensure that physics-
based predictions dominate when data-driven corrections are unreliable.

By design, the hybrid synthesis logic prioritizes physical plausibility, robustness,
and stable behavior over unconstrained accuracy improvements.

\subsection{Summary of Hybrid Rate Assembly}

The hybrid physics--ML rate synthesis combines deterministic physics-based predictions
with selectively applied machine-learning residual corrections. This structured
integration enables improved predictive accuracy while preserving interpretability,
physical consistency, and scalability across large well populations.


\section{Temporal Reconstruction of Historical Production Rates}
\label{sec:temporal-reconstruction}

In practical oil and gas operations, direct flow-rate measurements are often sparse,
irregular, or intermittently unavailable due to limited test frequency, sensor outages,
or operational constraints. As a result, historical production datasets frequently
contain missing values for oil, water, and gas rates. Beyond forward prediction, an
important capability of virtual flow metering systems is therefore the reconstruction
of temporally dense and physically consistent historical production profiles.

This section describes the methodology used to reconstruct historical production rates
using the proposed hybrid physics--machine learning framework.

\subsection{Problem Definition}

Let $t \in \mathcal{T}$ denote discrete time indices corresponding to available sensor
measurements. For each well, let
\begin{equation}
\mathbf{q}(t) = \bigl[q_o(t),\, q_w(t),\, q_g(t)\bigr]
\end{equation}
represent the true oil, water, and gas flow rates, which may be partially observed.
Define the set of time indices with missing production measurements as
\begin{equation}
\mathcal{T}_{\mathrm{miss}} =
\left\{
t \in \mathcal{T} \;:\;
\exists i \in \{o,w,g\} \;\text{such that}\; q_i(t) \;\text{is unobserved}
\right\}.
\end{equation}

The objective of temporal reconstruction is to estimate
\(
\hat{\mathbf{q}}(t)
\)
for all
\(
t \in \mathcal{T}_{\mathrm{miss}}
\),
using available sensor measurements and the trained hybrid model, while preserving
physical consistency across time.

\subsection{Distinction from Forecasting}

It is important to distinguish temporal reconstruction from forecasting. In the proposed framework, reconstruction operates exclusively on historical timestamps for which all independent variables (pressures, temperatures, choke settings, and lagged features) are available. 
No extrapolation beyond the observed time horizon is performed.

Formally, reconstruction estimates
\(
\hat{\mathbf{q}}(t)
\)
for
\(
t \le t_{\max}
\),
where $t_{\max}$ denotes the latest available measurement time, whereas forecasting
would require predicting
\(
\mathbf{q}(t)
\)
for
\(
t > t_{\max}
\).
The present work focuses solely on the former.

\subsection{Hybrid Inference for Missing Time Indices}

For each time index $t \in \mathcal{T}_{\mathrm{miss}}$, the hybrid inference procedure
described in Section~\ref{sec:hybrid-rate-synthesis} is applied. First, physics-based
predictions are generated:
\begin{equation}
\hat{\mathbf{q}}^{P}(t)
=
\bigl[
\hat{q}_o^{P}(t),\,
\hat{q}_w^{P}(t),\,
\hat{q}_g^{P}(t)
\bigr].
\end{equation}

Next, machine-learning residuals are inferred from the residual model using the same
feature construction employed during training:
\begin{equation}
\hat{\mathbf{r}}(t)
=
\bigl[
\hat{r}_o(t),\,
\hat{r}_w(t),\,
\hat{r}_g(t)
\bigr].
\end{equation}

The final reconstructed rates are obtained through phase-specific hybrid synthesis,
as defined by the equations in Section~\ref{sec:hybrid-rate-synthesis}:
\begin{equation}
\hat{\mathbf{q}}(t)
=
\mathcal{S}\!\left(
\hat{\mathbf{q}}^{P}(t),\,
\hat{\mathbf{r}}(t)
\right),
\end{equation}
where $\mathcal{S}(\cdot)$ denotes the hybrid physics--ML synthesis operator incorporating
log-space residual application, gating, and physical consistency constraints.

\subsection{Selective Reconstruction Logic}

In the implemented workflow, reconstruction is applied only to time indices for which
one or more dependent variables are missing. Let $\mathbf{y}(t)$ denote the vector of
measured rates when available. The reconstructed dataset $\tilde{\mathbf{q}}(t)$ is
defined as
\begin{equation}
\tilde{\mathbf{q}}(t)
=
\begin{cases}
\mathbf{y}(t), & \text{if all rate measurements are available}, \\[6pt]
\hat{\mathbf{q}}(t), & \text{if any rate measurement is missing}.
\end{cases}
\end{equation}

This selective replacement ensures that observed measurements are preserved exactly,
while the hybrid model is used only to fill missing values.

\subsection{Physical and Temporal Consistency}

The reconstruction procedure preserves physical consistency at each time step by
enforcing non-negativity of phase rates and liquid-rate consistency,
\begin{equation}
\hat{q}_L(t) = \hat{q}_o(t) + \hat{q}_w(t),
\end{equation}
with appropriate clipping applied as described previously. Temporal smoothness is
implicitly supported through the use of lagged features in the residual model, which
allow historical system states to influence reconstructed values without imposing
explicit smoothing constraints.

\subsection{Practical Implications}

Temporal reconstruction of historical production rates enables the creation of dense,
physically consistent time series suitable for production surveillance, performance
analysis, and downstream data-driven applications. By leveraging physics-based inference
as a stable baseline and applying machine-learning corrections selectively, the proposed
approach provides robust reconstruction under sparse and irregular measurement regimes
commonly encountered in field operations.

Overall, this capability extends the role of the proposed hybrid framework beyond
prediction, positioning it as a practical tool for historical data completion in
industrial virtual flow metering workflows.


\section{Model Integration}
\label{sec:model_integration}

\subsection{Step-by-Step Pipeline Summary}
\label{subsec:pipeline_summary}

\subsection{Computational Considerations}
\label{sec:computational_considerations}

The proposed hybrid modeling architecture combines a physics-based flow model with a data-driven residual learning component. While this integration improves predictive accuracy and generalization under sparse data conditions, it introduces several computational considerations related to efficiency, scalability, numerical stability, and deployment feasibility. These aspects are discussed below.

\subsubsection{Model Decomposition and Computational Load}

The hybrid architecture is deliberately decomposed into two stages: (i) a physics-based forward model and (ii) a machine learning model trained to predict residual errors. This decomposition allows the majority of physical computations to remain lightweight, relying on closed-form or low-dimensional parametric equations. Consequently, the computational complexity of the physics layer scales linearly with the number of samples, i.e.,
\begin{equation}
\mathcal{O}(N),
\end{equation}
where $N$ denotes the number of time steps.

The machine learning residual model operates on a reduced feature space, typically consisting of selected operational variables and physics-derived quantities. As a result, the hybrid model avoids the high-dimensional input spaces commonly encountered in end-to-end deep learning approaches, thereby reducing both training time and memory requirements.

\subsubsection{Parameter Calibration and Training Efficiency}

Parameter calibration in the physics layer is performed using nonlinear least squares optimization. Given the low dimensionality of the parameter vector, convergence is typically achieved within a small number of iterations. The computational cost of this calibration step is therefore modest and well-suited for per-well or batched multi-well calibration.

The residual learning component is trained after physics model calibration, which decouples physical parameter estimation from machine learning optimization. This sequential training strategy improves numerical stability and reduces the risk of ill-conditioned joint optimization. Furthermore, it allows reuse of calibrated physics parameters when transferring the model to new wells with limited data.

\subsubsection{Scalability Across Multiple Wells}

From a scalability perspective, the hybrid framework supports both per-well and global learning paradigms. Physics model calibration can be executed independently for each well, enabling embarrassingly parallel computation across wells. The residual model, on the other hand, may be trained globally using pooled data to capture cross-well patterns, with computational complexity depending on the chosen learning algorithm.

This modular design allows the framework to scale efficiently with the number of wells while maintaining manageable computational overhead, making it suitable for field-scale virtual flow metering applications.

\subsubsection{Numerical Stability and Robustness}

The inclusion of physics-based constraints inherently improves numerical stability by restricting predictions to physically plausible regimes. Additionally, normalization of inputs and careful handling of unit consistency (e.g., pressure in bar absolute and temperature in degrees Celsius) are employed to prevent numerical scaling issues during training.

Gating mechanisms used to combine physics predictions and residual corrections further enhance robustness by attenuating machine learning corrections under extrapolative or low-confidence conditions. This design reduces sensitivity to noisy measurements and mitigates the risk of unphysical predictions.

\subsubsection{Inference Time and Deployment Considerations}

At inference time, the hybrid model requires a single forward evaluation of the physics equations followed by a lightweight residual prediction. This results in low-latency predictions suitable for near–real-time applications. Compared to purely data-driven deep learning models, the proposed approach offers significantly reduced inference cost, making it amenable to deployment in production monitoring systems with limited computational resources.

Overall, the computational design of the hybrid model achieves a favorable balance between accuracy, interpretability, and efficiency, enabling practical deployment in real-world well surveillance and production optimization workflows.


\chapter{Experimentation}
\label{ch:experimentation}

This chapter describes the experimental design, data partitioning strategy, and evaluation protocols used to assess the performance of the proposed physics-informed residual learning virtual flow metering (VFM) model. The experiments are designed to provide a rigorous and realistic assessment of model accuracy, robustness, and generalization capability under conditions that closely resemble practical deployment in oil and gas production environments.

\section{Experimental Objectives}
\label{sec:experimentation_objectives}

The primary objectives of the experimental evaluation are to:
\begin{enumerate}
    \item Quantitatively evaluate the accuracy of oil, water, and gas flow rate predictions produced by the proposed hybrid physics--machine learning model using well-test measurements as reference data. Assess temporal generalization by evaluating model performance on unseen time periods.

    \item Quantify the performance gains achieved through hybrid residual correction model relative to physics-only model predictions across multiple wells.

    \item Compare the predictive performance of the proposed modeling approaches against MPFM estimates in order to assess the added value of the data-driven framework relative to field-measured metering solutions.

    \item Assess the spatial generalization capability of the proposed framework under realistic deployment conditions by evaluating model performance on previously unseen wells using a \abbr{Leave-One-Well-Out}{LOWO} strategy. For unseen wells, limited physics-model calibration is permitted, while the global machine-learning residual model is applied without retraining.

    \item Benchmark the predictive performance of the proposed hybrid framework against the results reported by \citep{nemoto2023cloud}, leveraging the fact that both studies are based on the same dataset. This comparison aims to contextualize the performance of the proposed approach relative to existing cloud-based virtual flow metering methodologies under similar data.
\end{enumerate}


\section{Experimental Data Partitioning}

To support robust model development and unbiased evaluation, the available dataset is partitioned into training, validation, and test subsets on a per-well basis. The partitioning strategy is designed to (i) preserve temporal ordering within each subset, (ii) ensure representation of all wells across splits, and (iii) maintain sufficient sample sizes for model calibration and evaluation.

For each well, the data are randomly assigned to the three subsets using fixed proportions:
\begin{itemize}
    \item \textbf{Training set}: 70\% of the data
    \item \textbf{Validation set}: 10\% of the data
    \item \textbf{Test set}: 20\% of the data
\end{itemize}

Although the assignment is random at the row level, the original temporal order of observations is preserved within each subset. This ensures that sequential dependencies are not violated during model training or evaluation, while still allowing sufficient variability across the splits.

To guarantee statistical reliability, minimum sample size constraints are enforced for each well:
\begin{itemize}
    \item At least 20 samples in the training set
    \item At least 5 samples in the validation set
    \item At least 5 samples in the test set
\end{itemize}

If these constraints are not satisfied for any well, the split is discarded and re-sampled. This process is repeated up to a maximum number of attempts until a valid split satisfying all constraints is obtained. The final training, validation, and test datasets are formed by concatenating the corresponding subsets across all wells.

This partitioning approach ensures that all wells contribute data to each experimental phase, avoids data leakage between training and testing, and reflects realistic operational conditions in which future observations are evaluated using models trained on historical data.


\section{Production-Mimicking Experimental Workflow}

All experiments are conducted using a production-mimicking workflow. Physics models calibrated during the training phase remain fixed throughout the experimentation, and the globally trained machine-learning residual model is applied strictly in inference mode. No retraining or parameter updates are performed during testing.

Let $t$ denote the prediction time index. At each time step $t$, only measurements available up to that instant are used for prediction, thereby emulating real-time or near-real-time virtual flow metering operation.

\section{Experimental Model Configurations}

Two experimental model configurations are considered:

\begin{enumerate}
    \item \textbf{Physics-only configuration}:  
    Flow rates are estimated using a physics-based model calibrated on a per-well basis. For unseen wells, limited calibration of well-specific physics parameters is performed, while no data-driven correction is applied.

    \item \textbf{Physics-informed hybrid configuration}:  
    Physics-based predictions are corrected using a globally trained machine-learning residual model operating in log-space, with regime-aware gating and physical constraints enforced. For unseen wells, only the physics-model parameters are calibrated, and the residual model is applied without retraining.
\end{enumerate}

Comparing these configurations allows the incremental contribution of residual learning to be quantified.

\section{Evaluation Metrics}
\label{sec:evaluation_metrics}

Model performance is quantified using a set of complementary metrics that capture absolute error, relative error, systematic bias, and goodness of fit. Let $y_i$ denote the measured value of a given flow rate at time index $i$, $\hat{y}_i$ the corresponding model prediction, and $N$ the number of evaluated samples.

\subsection{Coefficient of Determination}

The coefficient of determination ($R^2$) is defined as:
\begin{equation}
R^2 = 1 - \frac{\sum_{i=1}^{N} (y_i - \hat{y}_i)^2}
{\sum_{i=1}^{N} (y_i - \bar{y})^2},
\end{equation}
where $\bar{y}$ denotes the mean of the measured values. The $R^2$ metric is unitless and indicates the proportion of variance in the measured data explained by the model.

\subsection{Mean Absolute Error}

The mean absolute error (MAE) is defined as:
\begin{equation}
\text{MAE} = \frac{1}{N} \sum_{i=1}^{N} \left| y_i - \hat{y}_i \right|.
\end{equation}
MAE is expressed in the same physical units as the predicted variable (e.g., Sm$^3$/h) and provides an intuitive measure of the average magnitude of prediction error.

\subsection{Root Mean Squared Error}

The root mean squared error (RMSE) is defined as:
\begin{equation}
\text{RMSE} =
\sqrt{\frac{1}{N} \sum_{i=1}^{N} (y_i - \hat{y}_i)^2 }.
\end{equation}
RMSE is expressed in the same physical units as the predicted variable (e.g., Sm$^3$/h). Compared to MAE, RMSE penalizes larger errors more strongly and is therefore more sensitive to outliers.

\subsection{Mean Absolute Percentage Error}

The mean absolute percentage error (MAPE), also referred to as the mean absolute relative error (MARE), is defined as:
\begin{equation}
\text{MAPE} = \frac{100}{N}
\sum_{i=1}^{N}
\left| \frac{y_i - \hat{y}_i}{y_i} \right|.
\end{equation}
MAPE is a dimensionless metric reported as a percentage (\%). It quantifies the average relative magnitude of prediction error. For low-rate regimes, particularly for water production, MAPE may exhibit inflated values and is therefore interpreted with caution.

\subsection{Mean Percentage Error}

The mean percentage error (MPE), also referred to as the signed mean relative error or average discrepancy, is defined as:
\begin{equation}
\text{MPE} = \frac{100}{N}
\sum_{i=1}^{N}
\frac{\hat{y}_i - y_i}{y_i}.
\end{equation}
MPE is a dimensionless metric reported as a percentage (\%). It captures systematic prediction bias, with positive values indicating overprediction and negative values indicating underprediction.

\section{Evaluation of Derived Production Ratios}

In addition to individual phase flow rates, the experiments evaluate derived production ratios, including the water--gas ratio (WGR) and gas--oil ratio (GOR), defined as:
\begin{equation}
\text{wgr} = \frac{q_w}{q_g},
\qquad
\text{gor} = \frac{q_g}{q_o},
\end{equation}
where $q_o$, $q_w$, and $q_g$ denote oil, water, and gas flow rates, respectively.

These ratios are computed from both measured and predicted flow rates, and the same evaluation metrics are applied. Data points associated with very low denominator values are excluded to avoid numerical instability.

\section{Cross-Well Temporal Generalization Evaluation Setup}
\label{sec:temporal_setup}

The cross-well temporal generalization evaluation is designed to assess the ability of the proposed modeling framework to produce accurate flow-rate predictions on unseen time periods while remaining within the same set of wells. This evaluation reflects a common production monitoring scenario in which historical data are available for a well, and the objective is to predict flow rates under future operating conditions without retraining the model.

For each well, the available time-series data are partitioned into training and testing subsets using a temporally ordered split. Earlier observations are used for physics-model calibration and, where applicable, training of the global machine-learning residual model, while later observations are held out exclusively for evaluation. This strategy preserves the natural temporal ordering of the data and prevents information leakage from future operating conditions into the training process.

The physics-only and hybrid model configurations are evaluated under identical temporal splits. Physics-based model parameters are calibrated using the training portion of the data for each well. The machine-learning residual model is trained globally using training data from all wells and is applied during inference without retraining. Multiphase flow meter MPFM estimates are evaluated on the same test intervals for fair comparison.

Model performance is quantified using well-test measurements as the reference standard and evaluated using the evaluation metrics defined in Section~\ref{sec:evaluation_metrics} for oil, water, and gas flow rates. 
Reporting results on a per-well basis enables systematic assessment of temporal robustness under evolving operating conditions.

\section{Spatial Generalization Evaluation Setup}
\label{sec:lowo_setup}

The spatial generalization evaluation assesses the ability of the proposed modeling framework to transfer predictive capability to a previously unseen well. 
This evaluation is conducted using a targeted Leave-One-Well-Out (LOWO) strategy, designed to emulate a realistic deployment scenario in which a trained model is applied to a new well with limited or no prior exposure during training.

In this setup, the well with the largest number of historical observations (W10) is excluded entirely from model development and used exclusively for evaluation, while the remaining wells constitute the training set. The machine-learning residual model is trained globally using data from the training wells and is kept fixed during evaluation on the unseen well. For the held-out well, limited calibration of well-specific physics parameters is performed using a small subset of available data to ensure physical consistency, while retraining of the residual model is intentionally avoided.

This experimental design serves as a conservative stress test of spatial generalization. By excluding the most data-rich well from training, the evaluation prevents dominance of a single well during residual learning and ensures that model performance on W10 reflects genuine cross-well transferability rather than implicit memorization driven by data availability. The setup therefore provides a stringent assessment of whether the proposed hybrid framework can generalize to a well with complex and well-sampled production behavior based solely on information learned from other wells.

Both physics-only and hybrid model configurations are evaluated on the held-out well, and their performance is compared against MPFM estimates using well-test measurements as the reference standard. Model performance is quantified using the evaluation metrics defined in Section~\ref{sec:evaluation_metrics} for oil, water, and gas flow rates.


\section{Summary of Experimental Design}

The experimental design emphasizes realism, reproducibility, and practical relevance. By enforcing temporal separation, preventing data leakage, and mimicking production deployment conditions, the experiments provide a comprehensive and unbiased assessment of the proposed virtual flow metering model. The results presented in the following chapter build upon this experimental framework to demonstrate the effectiveness of physics-informed residual learning for multiphase flow rate estimation in oil and gas production wells.

\begin{table}[h!]
\centering
\caption{Performance metrics used for model evaluation}
\label{tab:metrics}
\begin{tabularx}{\textwidth}{l X l}
\hline
\textbf{Metric} & \textbf{Description} & \textbf{Units} \\
\hline
$R^2$ &
Coefficient of determination measuring goodness of fit &
unitless \\

MAE &
Mean Absolute Error, average magnitude of prediction error &
Same as target (e.g., Sm$^3$/h) \\

RMSE &
Root Mean Squared Error, penalizing large deviations &
Same as target (e.g., Sm$^3$/h) \\

MAPE / MARE &
Mean absolute relative prediction error &
\% \\

MPE / MRE &
Signed mean relative error (average discrepancy) &
\% \\
\hline
\end{tabularx}
\end{table}


\chapter{Results and Analysis}

This chapter presents the experimental results and analysis of the proposed hybrid physics--machine learning virtual flow metering framework. 

\section{Temporal Generalization Performance}

This section presents per-well prediction accuracy on unseen time periods, based on the temporal generalization evaluation setup described in Section~\ref{sec:temporal_setup}.

\begin{table}[!htbp]
\centering
\caption{Prediction performance comparison for oil ($q_o$), water ($q_w$), and gas ($q_g$) rates across all wells.}
\label{tab:rate_scores_by_phase}
\small
\setlength{\tabcolsep}{4pt}
\begin{tabular}{ll|ccc|ccc|ccc}
\toprule
 &  & \multicolumn{9}{c}{Phase} \\
\cmidrule(lr){3-11}
Well & Model
& \multicolumn{3}{c}{Oil ($q_o$)}
& \multicolumn{3}{c}{Water ($q_w$)}
& \multicolumn{3}{c}{Gas ($q_g$)} \\
\cmidrule(lr){3-5} \cmidrule(lr){6-8} \cmidrule(lr){9-11}
 & 
 & $R^2$ & MAE & RMSE
 & $R^2$ & MAE & RMSE
 & $R^2$ & MAE & RMSE \\
\midrule

% ===================== W06 =====================
\multirow{3}{*}{W06}
& Physics & 0.70 & 13.30 & 17.49 & 0.05 & 1.04 & 1.28 & 0.52 & 2311.07 & 2828.52 \\
& Hybrid  & 0.82 & 9.81  & 13.74 & -1.27 & 1.48 & 1.98 & 0.85 & 1233.78 & 1563.34 \\
& MPFM    & 0.95 & 5.93  & 7.26  & -0.93 & 1.38 & 1.82 & 0.96 & 570.81  & 801.07  \\
\midrule

% ===================== W08 =====================
\multirow{3}{*}{W08}
& Physics & 0.15 & 8.98 & 10.83 & 0.75 & 6.28 & 7.23 & 0.36 & 1155.35 & 1548.24 \\
& Hybrid  & 0.12 & 7.63 & 10.96 & 0.71 & 6.08 & 7.78 & 0.35 & 1162.30 & 1563.73 \\
& MPFM    & 0.83 & 3.99 & 4.88  & 0.95 & 2.73 & 3.35 & 0.67 & 751.99  & 1117.27 \\
\midrule

% ===================== W10 =====================
\multirow{3}{*}{W10}
& Physics & 0.45 & 19.38 & 24.82 & -3.85 & 1.96 & 3.05 & 0.89 & 1197.22 & 1570.05 \\
& Hybrid  & 0.86 & 9.91  & 12.59 & -1.15 & 1.67 & 2.03 & 0.92 & 1060.62 & 1338.47 \\
& MPFM    & 0.98 & 3.11  & 4.53  & -2.99 & 2.31 & 2.80 & 0.92 & 530.72  & 1365.75 \\
\midrule

% ===================== W11 =====================
\multirow{3}{*}{W11}
& Physics & 0.03 & 36.80 & 47.13 & 0.01 & 5.05 & 6.07 & 0.55 & 2623.74 & 3290.68 \\
& Hybrid  & 0.03 & 34.26 & 46.98 & 0.49 & 3.17 & 4.38 & 0.72 & 2235.44 & 2602.49 \\
& MPFM    & 0.78 & 8.63  & 22.06 & 0.54 & 2.94 & 4.57 & 0.94 & 715.94  & 1228.54 \\
\midrule

% ===================== W15 =====================
\multirow{3}{*}{W15}
& Physics & 0.32 & 9.75 & 12.15 & 0.21 & 1.31 & 1.69 & 0.80 & 1013.10 & 1337.01 \\
& Hybrid  & 0.84 & 4.88 & 5.87  & -0.12 & 1.23 & 2.02 & 0.74 & 1132.46 & 1516.85 \\
& MPFM    & 0.92 & 2.42 & 4.04  & 0.02 & 1.21 & 1.89 & 0.73 & 704.36  & 1562.90 \\
\midrule

% ===================== W18 =====================
\multirow{3}{*}{W18}
& Physics & 0.08 & 22.54 & 27.79 & 0.55 & 13.82 & 24.96 & 0.34 & 2632.44 & 3712.13 \\
& Hybrid  & 0.72 & 11.70 & 15.43 & 0.66 & 10.31 & 21.69 & 0.52 & 2402.84 & 3174.49 \\
& MPFM    & 0.68 & 8.60  & 16.54 & 0.50 & 9.18  & 26.27 & 0.40 & 1312.41 & 3518.69 \\
\midrule

% ===================== W19 =====================
\multirow{3}{*}{W19}
& Physics & -0.32 & 19.09 & 24.68 & 0.64 & 3.63 & 4.86 & 0.44 & 2694.17 & 3023.05 \\
& Hybrid  & 0.65  & 10.06 & 12.65 & 0.78 & 2.52 & 3.78 & 0.60 & 2018.05 & 2553.24 \\
& MPFM    & 0.96  & 2.67  & 4.31  & 0.89 & 1.49 & 2.81 & 0.95 & 594.66  & 971.11  \\
\bottomrule
\end{tabular}
\end{table}


\begin{figure}[!htb]
    \centering

    \begin{subfigure}[t]{0.48\textwidth}
        \uomFig
        {W06 gas rate temporal generalization performance}
        {fig:w06_qg_model_performance}
        {\includegraphics[width=\textwidth]{images/results/W06_qg_hybrid.png}}{}
    \end{subfigure}
    \hfill
    \begin{subfigure}[t]{0.48\textwidth}
        \uomFig
        {W06 oil rate temporal generalization performance}
        {fig:w06_qo_model_performance}
        {\includegraphics[width=\textwidth]{images/results/W06_qo_hybrid.png}}{}
    \end{subfigure}

    \vspace{0.5em}

    \begin{subfigure}[t]{0.48\textwidth}
        \uomFig
        {W06 water rate temporal generalization performance}
        {fig:w06_qw_model_performance}
        {\includegraphics[width=\textwidth]{images/results/W06_qw_hybrid.png}}{}
    \end{subfigure}

    \caption{Temporal generalization performance of the hybrid model for W06.
    The plots show predicted and measured gas, oil, and water rates over time.}
    \label{fig:w06_temporal_generalization}
\end{figure}


\begin{figure}[!htb]
    \centering

    \begin{subfigure}[t]{0.48\textwidth}
        \uomFig
        {W08 gas rate temporal generalization performance}
        {fig:W08_qg_model_performance}
        {\includegraphics[width=\textwidth]{images/results/W08_qg_hybrid.png}}{}
    \end{subfigure}
    \hfill
    \begin{subfigure}[t]{0.48\textwidth}
        \uomFig
        {W08 oil rate temporal generalization performance}
        {fig:W08_qo_model_performance}
        {\includegraphics[width=\textwidth]{images/results/W08_qo_hybrid.png}}{}
    \end{subfigure}

    \vspace{0.5em}

    \begin{subfigure}[t]{0.48\textwidth}
        \uomFig
        {W08 water rate temporal generalization performance}
        {fig:W08_qw_model_performance}
        {\includegraphics[width=\textwidth]{images/results/W08_qw_hybrid.png}}{}
    \end{subfigure}

    \caption{Temporal generalization performance of the hybrid model for W08.
    The plots show predicted and measured gas, oil, and water rates over time.}
    \label{fig:W08_temporal_generalization}
\end{figure}


\begin{figure}[!htb]
    \centering

    \begin{subfigure}[t]{0.48\textwidth}
        \uomFig
        {W10 gas rate temporal generalization performance}
        {fig:w10_qg_model_performance}
        {\includegraphics[width=\textwidth]{images/results/W10_qg_hybrid.png}}{}
    \end{subfigure}
    \hfill
    \begin{subfigure}[t]{0.48\textwidth}
        \uomFig
        {W10 oil rate temporal generalization performance}
        {fig:w10_qo_model_performance}
        {\includegraphics[width=\textwidth]{images/results/W10_qo_hybrid.png}}{}
    \end{subfigure}

    \vspace{0.5em}

    \begin{subfigure}[t]{0.48\textwidth}
        \uomFig
        {W10 water rate temporal generalization performance}
        {fig:w10_qw_model_performance}
        {\includegraphics[width=\textwidth]{images/results/W10_qw_hybrid.png}}{}
    \end{subfigure}

    \caption{Temporal generalization performance of the hybrid model for W10.
    The plots show predicted and measured gas, oil, and water rates over time.}
    \label{fig:w10_temporal_generalization}
\end{figure}

\begin{figure}[!htb]
    \centering

    \begin{subfigure}[t]{0.48\textwidth}
        \uomFig
        {W11 gas rate temporal generalization performance}
        {fig:W11_qg_model_performance}
        {\includegraphics[width=\textwidth]{images/results/W11_qg_hybrid.png}}{}
    \end{subfigure}
    \hfill
    \begin{subfigure}[t]{0.48\textwidth}
        \uomFig
        {W11 oil rate temporal generalization performance}
        {fig:W11_qo_model_performance}
        {\includegraphics[width=\textwidth]{images/results/W11_qo_hybrid.png}}{}
    \end{subfigure}

    \vspace{0.5em}

    \begin{subfigure}[t]{0.48\textwidth}
        \uomFig
        {W11 water rate temporal generalization performance}
        {fig:W11_qw_model_performance}
        {\includegraphics[width=\textwidth]{images/results/W11_qw_hybrid.png}}{}
    \end{subfigure}

    \caption{Temporal generalization performance of the hybrid model for W11.
    The plots show predicted and measured gas, oil, and water rates over time.}
    \label{fig:W11_temporal_generalization}
\end{figure}


\begin{figure}[!htb]
    \centering

    \begin{subfigure}[t]{0.48\textwidth}
        \uomFig
        {W15 gas rate temporal generalization performance}
        {fig:W15_qg_model_performance}
        {\includegraphics[width=\textwidth]{images/results/W15_qg_hybrid.png}}{}
    \end{subfigure}
    \hfill
    \begin{subfigure}[t]{0.48\textwidth}
        \uomFig
        {W15 oil rate temporal generalization performance}
        {fig:W15_qo_model_performance}
        {\includegraphics[width=\textwidth]{images/results/W15_qo_hybrid.png}}{}
    \end{subfigure}

    \vspace{0.5em}

    \begin{subfigure}[t]{0.48\textwidth}
        \uomFig
        {W15 water rate temporal generalization performance}
        {fig:W15_qw_model_performance}
        {\includegraphics[width=\textwidth]{images/results/W15_qw_hybrid.png}}{}
    \end{subfigure}

    \caption{Temporal generalization performance of the hybrid model for W15.
    The plots show predicted and measured gas, oil, and water rates over time.}
    \label{fig:W15_temporal_generalization}
\end{figure}


\begin{figure}[!htb]
    \centering

    \begin{subfigure}[t]{0.48\textwidth}
        \uomFig
        {W18 gas rate temporal generalization performance}
        {fig:w18_qg_model_performance}
        {\includegraphics[width=\textwidth]{images/results/W18_qg_hybrid.png}}{}
    \end{subfigure}
    \hfill
    \begin{subfigure}[t]{0.48\textwidth}
        \uomFig
        {W18 oil rate temporal generalization performance}
        {fig:w18_qo_model_performance}
        {\includegraphics[width=\textwidth]{images/results/W18_qo_hybrid.png}}{}
    \end{subfigure}

    \vspace{0.5em}

    \begin{subfigure}[t]{0.48\textwidth}
        \uomFig
        {W18 water rate temporal generalization performance}
        {fig:w18_qw_model_performance}
        {\includegraphics[width=\textwidth]{images/results/W18_qw_hybrid.png}}{}
    \end{subfigure}

    \caption{Temporal generalization performance of the hybrid model for W18.
    The plots show predicted and measured gas, oil, and water rates over time.}
    \label{fig:w18_temporal_generalization}
\end{figure}


\begin{figure}[!htb]
    \centering

    \begin{subfigure}[t]{0.48\textwidth}
        \uomFig
        {W19 gas rate temporal generalization performance}
        {fig:W19_qg_model_performance}
        {\includegraphics[width=\textwidth]{images/results/W19_qg_hybrid.png}}{}
    \end{subfigure}
    \hfill
    \begin{subfigure}[t]{0.48\textwidth}
        \uomFig
        {W19 oil rate temporal generalization performance}
        {fig:W19_qo_model_performance}
        {\includegraphics[width=\textwidth]{images/results/W19_qo_hybrid.png}}{}
    \end{subfigure}

    \vspace{0.5em}

    \begin{subfigure}[t]{0.48\textwidth}
        \uomFig
        {W19 water rate temporal generalization performance}
        {fig:W19_qw_model_performance}
        {\includegraphics[width=\textwidth]{images/results/W19_qw_hybrid.png}}{}
    \end{subfigure}

    \caption{Temporal generalization performance of the hybrid model for W19.
    The plots show predicted and measured gas, oil, and water rates over time.}
    \label{fig:W19_temporal_generalization}
\end{figure}


\section{Spatial Generalization Performance}

This section reports model performance on previously unseen wells, based on the LOWO evaluation protocol defined in Section~\ref{sec:lowo_setup}.

\begin{table}[htbp]
\centering
\caption{Performance comparison of physics-only and hybrid models for Well W10
under the Leave-One-Well-Out (LOWO) evaluation strategy.
Absolute error metrics are emphasized for robustness under low-rate regimes.}
\label{tab:lowo_w10_performance_clean}
\resizebox{\textwidth}{!}{
\begin{tabular}{llccc}
\toprule
\textbf{Rate} & \textbf{Model} & $R^2$ & MAE & RMSE \\
\midrule
\multirow{2}{*}{Oil rate ($q_o$)}
& Physics-only & 0.476 & 17.67 & 23.27 \\
& Hybrid       & \textbf{0.654} & \textbf{15.48} & \textbf{18.91} \\
\midrule
\multirow{2}{*}{Water rate ($q_w$)}
& Physics-only & 0.640 & 3.27 & 6.18 \\
& Hybrid       & \textbf{0.717} & \textbf{3.10} & \textbf{5.48} \\
\midrule
\multirow{2}{*}{Gas rate ($q_g$)}
& Physics-only & \textbf{0.691} & \textbf{1649.56} & \textbf{2271.27} \\
& Hybrid       & 0.133 & 3120.42 & 3806.95 \\
\midrule
\multirow{2}{*}{WGR}
& Physics-only & 0.456 & $2.88\times10^{-4}$ & $6.64\times10^{-4}$ \\
& Hybrid       & \textbf{0.555} & \textbf{$2.61\times10^{-4}$} & \textbf{$6.01\times10^{-4}$} \\
\bottomrule
\end{tabular}
}
\end{table}


\section{Code Availability and Reproducibility}
\label{sec:reproducibility}

All code implementations, experimental results, supporting documentation, and supplementary literature associated with this research are publicly available in the GitHub repository \url{https://github.com/pcperera/vfm}. 
This repository is maintained by the author and serves as the primary reference for the computational aspects of this thesis.

The repository includes the complete implementation of the proposed physics-informed residual learning framework for virtual flow metering, along with data preprocessing routines, model training and evaluation scripts, and visualization utilities used to generate the results presented in this thesis. Where applicable, documentation is provided to facilitate reproducibility and to enable further extension of the proposed methods.

\chapter{Discussion}

\chapter{Future Work}
\label{chap:future-work}

The physics-informed residual learning framework developed in this study demonstrates promising performance for virtual flow metering under sparse and irregular measurement conditions. While the proposed approach addresses several practical limitations of conventional flow metering techniques, there remain multiple avenues for further research and enhancement. This chapter outlines potential directions for future work aimed at improving model fidelity, robustness, scalability, and industrial applicability.

\section{Extension of Physics-Based Modeling}

The current physics layer employs a simplified steady-state formulation to model liquid and gas production based on pressure, temperature, and choke measurements. Future work could extend this formulation to incorporate more detailed multiphase flow physics, including frictional pressure losses, phase slip effects, and temperature-dependent fluid properties. Incorporating transient flow equations or quasi-steady corrections may further improve performance under highly unsteady operating conditions, such as during choke adjustments or flow instabilities.

In addition, the physics model could be enhanced by integrating well-specific geological and completion parameters, such as tubing roughness, completion depth, or reservoir connectivity, thereby reducing the reliance on purely data-driven corrections.

\section{Advanced Residual Learning Architectures}

The residual learning component in the current framework is implemented using tree-based ensemble models for computational efficiency and robustness. Future research could investigate alternative residual learning architectures, including recurrent neural networks or attention-based temporal models, to better capture time-dependent error structures. Such models may be particularly beneficial for wells exhibiting strong temporal correlations or delayed flow responses.

Hybrid residual architectures that combine lightweight temporal models with regime-aware gating mechanisms could further improve accuracy while maintaining computational tractability.

\section{Dynamic Regime Detection and Adaptive Gating}

In the present framework, residual corrections are selectively applied using predefined gating rules based on physical heuristics such as water cut thresholds and residual magnitude. Future work could focus on developing adaptive or data-driven regime detection mechanisms that automatically identify flow regimes and dynamically adjust correction strategies. Probabilistic gating or soft-switching approaches may provide smoother transitions between physics-dominated and data-driven predictions.

\section{Uncertainty Quantification and Confidence Estimation}

The current model produces point estimates of flow rates without explicit uncertainty bounds. Future research could incorporate uncertainty quantification techniques, such as Bayesian residual learning or ensemble-based confidence estimation, to provide probabilistic flow predictions. Quantifying predictive uncertainty is critical for operational decision-making, production forecasting, and risk assessment in field-scale applications.

\section{Validation Across Diverse Field Conditions}

Future studies should validate the proposed framework across a wider range of reservoirs, well types, and production conditions. Applying the model to different asset classes, including gas-dominated wells, mature waterfloods, and unconventional reservoirs, would further assess its generalizability and robustness. Such validation would strengthen confidence in the framework’s applicability beyond the specific datasets considered in this study.

Overall, these future research directions aim to enhance the physical realism, adaptability, and industrial relevance of physics-informed residual learning approaches for virtual flow metering, building upon the foundational architecture developed in this work.

\begin{references}
    \bibliography{references} % argument is your bibliography database(s)
\end{references}

\end{document}
